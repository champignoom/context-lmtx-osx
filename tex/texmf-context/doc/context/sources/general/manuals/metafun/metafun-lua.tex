% language=us runpath=texruns:manuals/metafun

% this is an extension of about-lua

\startcomponent mfun-lua

\environment metafun-environment

\startchapter[title={Lua}]

\index{\LUA}

\startintro

Already for some years I have been wondering how it would be if we could escape
to \LUA\ inside \METAPOST, or in practice, use \MPLIB\ in \LUATEX. The idea is
simple: embed \LUA\ code in a \METAPOST\ file that gets run as soon as it's seen.
In case you wonder why \LUA\ code makes sense, imagine generating graphics using
external data. The capabilities of \LUA\ to deal with that is more flexible and
advanced than in \METAPOST. Of course we could generate a \METAPOST\ definition
of a graphic from data but often it makes more sense to do the reverse. I finally
found time and reason to look into this and in the following sections I will
describe how it's done.

\blank

{\bi The \LUA\ interface in \MKIV\ is way more limited than in \LMTX\ and new
features will only show up in \LMTX, simply because the \METAPOST\ library in
\LUAMETATEX\ is more powerful.}

\stopintro

% \startsection[title=Introduction]
%
% \stopsection

\startsection[title=The basics]

The approach is comparable to \LUATEX's \type {\directlua}. That primitive can be
used to execute \LUA\ code and in combination with \type {tex.print} we can pipe
back strings into the \TEX\ input stream. There a complication is that that we
have to be able to operate under different so called catcode regimes: the meaning
of characters can differ per regime. We also have to deal with line endings in
special ways as they relate to paragraphs and such. In \METAPOST\ we don't have
that complication so getting back input into the \METAPOST\ input, we can do so
with simple strings. For that a mechanism similar to \type {scantokens} can be
used. That way we can return anything (including nothing) as long as \METAPOST\
can interpret it and as long as it fulfils the expectations.

\starttyping
numeric n ; n := scantokens("123.456") ;
\stoptyping

A script is run as follows:

\starttyping
numeric n ; n := runscript("return '123.456'") ;
\stoptyping

This primitive doesn't have the word \type {lua} in its name so in principle any
wrapper around the library can use it as hook. In the case of \LUATEX\ the script
language is of course \LUA. At the \METAPOST\ end we only expect a string. How
that string is constructed is completely up to the \LUA\ script. In fact, the
user is completely free to implement the runner any way she or he wants, like:

\starttyping
local function scriptrunner(code)
    local f = loadstring(code)
    if f then
        return tostring(f())
    else
        return ""
    end
end
\stoptyping

This is hooked into an instance as follows:

\starttyping
local m = mplib.new {
    ...
    run_script = scriptrunner,
    ...
}
\stoptyping

Now, beware, this is not the \CONTEXT\ way. We provide print functions and other
helpers, which we will explain in the next section.

\stopsection

\startsection[title=Helpers]

After I got this feature up and running I played a bit with possible interfaces
at the \CONTEXT\ (read: \METAFUN) end and ended up with a bit more advanced runner
where no return value is used. The runner is wrapped in the \type {lua} macro.

\startbuffer
numeric n ; n := lua("mp.print(12.34567)") ;
draw textext(n) xsized 4cm withcolor darkred ;
\stopbuffer

\typebuffer

This renders as:

\startlinecorrection[blank]
\processMPbuffer
\stoplinecorrection

In case you wonder how efficient calling \LUA\ is, don't worry: it's fast enough,
especially if you consider suboptimal \LUA\ code and the fact that we switch
between machineries.

\startbuffer
draw image (
    lua("statistics.starttiming()") ;
    for i=1 upto 10000 : draw
        lua("mp.pair(math.random(-200,200),math.random(-50,50))") ;
    endfor ;
    setbounds currentpicture to fullsquare xyscaled (400,100) ;
    lua("statistics.stoptiming()") ;
) withcolor darkyellow withpen pencircle scaled 1 ;
draw textext(lua("mp.print(statistics.elapsedtime())"))
    ysized 50 withcolor darkred ;
\stopbuffer

\typebuffer

Here the line:

\starttyping
draw lua("mp.pair(math.random(-200,200),math.random(-50,50))") ;
\stoptyping

effectively becomes (for instance):

\starttyping
draw scantokens "(25,40)" ;
\stoptyping

which in turn becomes:

\starttyping
draw scantokens (25,40) ;
\stoptyping

The same happens with this:

\starttyping
draw textext(lua("mp.print(statistics.elapsedtime())")) ...
\stoptyping

This becomes for instance:

\starttyping
draw textext(scantokens "1.23") ...
\stoptyping

and therefore:

\starttyping
draw textext(1.23) ...
\stoptyping

We can use \type {mp.print} here because the \type {textext} macro can deal with
numbers. The next also works:

\starttyping
draw textext(lua("mp.quoted(statistics.elapsedtime())")) ...
\stoptyping

Now we get (in \METAPOST\ speak):

\starttyping
draw textext(scantokens (ditto & "1.23" & ditto) ...
\stoptyping

Here \type {ditto} represents the double quotes that mark a string. Of course,
because we pass the strings directly to \type {scantokens}, there are no outer
quotes at all, but this is how it can be simulated. In the end we have:

\starttyping
draw textext("1.23") ...
\stoptyping

What you use, \type {mp.print} or \type {mp.quoted} depends on what the expected
code is: an assignment to a numeric can best be a number or an expression
resulting in a number.

This graphic becomes:

\startlinecorrection[blank]
\processMPbuffer
\stoplinecorrection

The runtime on my current machine is some 0.25 seconds without and 0.12 seconds
with caching. But to be honest, speed is not really a concern here as the amount
of complex \METAPOST\ graphics can be neglected compared to extensive node list
manipulation. With \LUAJITTEX\ generating the graphic takes 15\% less time.

\startbuffer
numeric n ; n := lua("mp.print(1) mp.print('+') mp.print(2)") ;
draw textext(n) xsized 1cm withcolor darkred ;
\stopbuffer

The three print command accumulate their arguments:

\typebuffer

As expected we get:

\startlinecorrection[blank]
\processMPbuffer
\stoplinecorrection

\startbuffer
numeric n ; n := lua("mp.print(1,'+',2)") ;
draw textext(n) xsized 1cm withcolor darkred ;
\stopbuffer

Equally valid is:

\typebuffer

This gives the same result:

\startlinecorrection[blank]
\processMPbuffer
\stoplinecorrection

Of course all kind of action can happen between the prints. It is also legal to
have nothing returned as could be seen in the 10.000 dot example: there the timer
related code returns nothing so effectively we have \type {scantokens("")}. Another
helper is \type {mp.quoted}, as in:

\startbuffer
draw
    textext(lua("mp.quoted('@0.3f'," & decimal 1.234 & ")"))
    withcolor darkred ;
\stopbuffer

\typebuffer

This typesets \processMPbuffer. Watch the \type {@}. When no percent character is
found in the format specifier, we assume that an \type {@} is used instead.

\startbuffer
\startluacode
table.save("demo-data.lua",
    {
        { 1, 2 }, { 2, 4 }, { 3, 3 }, { 4, 2 },
        { 5, 2 }, { 6, 3 }, { 7, 4 }, { 8, 1 },
    }
)
\stopluacode
\stopbuffer

But, the real benefit of embedded \LUA\ is when we deal with data that is stored
at the \LUA\ end. First we define a small dataset:

\typebuffer

\getbuffer

There are several ways to deal with this table. I will show clumsy as well as
better looking ways.

\startbuffer
lua("MP.data = table.load('demo-data.lua')") ;
numeric n ;
lua("mp.print('n := ',\#MP.data)") ;
for i=1 upto n :
    drawdot
        lua("mp.pair(MP.data[" & decimal i & "])") scaled cm
        withpen pencircle scaled 2mm
        withcolor darkred ;
endfor ;
\stopbuffer

\typebuffer

Here we load a \LUA\ table and assign the size to a \METAPOST\ numeric. Next we
loop over the table entries and draw the coordinates.

\startlinecorrection[blank]
\processMPbuffer
\stoplinecorrection

We will stepwise improve this code. In the previous examples we omitted wrapper
code but here we show it:

\startbuffer
\startluacode
    MP.data = table.load('demo-data.lua')
    function MP.n()
        mp.print(#MP.data)
    end
    function MP.dot(i)
        mp.pair(MP.data[i])
    end
\stopluacode

\startMPcode
    numeric n ; n := lua("MP.n()") ;
    for i=1 upto n :
        drawdot
            lua("MP.dot(" & decimal i & ")") scaled cm
            withpen pencircle scaled 2mm
            withcolor darkred ;
    endfor ;
\stopMPcode
\stopbuffer

\typebuffer

So, we create a few helpers in the \type {MP} table. This table is predefined so
normally you don't need to define it. You may however decide to wipe it clean.

\startlinecorrection[blank]
\getbuffer
\stoplinecorrection

You can decide to hide the data:

\startbuffer
\startluacode
    local data = { }
    function MP.load(name)
        data = table.load(name)
    end
    function MP.n()
        mp.print(#data)
    end
    function MP.dot(i)
        mp.pair(data[i])
    end
\stopluacode
\stopbuffer

\typebuffer \getbuffer

It is possible to use less \LUA, for instance in:

\startbuffer
\startluacode
    local data = { }
    function MP.loaded(name)
        data = table.load(name)
        mp.print(#data)
    end
    function MP.dot(i)
        mp.pair(data[i])
    end
\stopluacode

\startMPcode
    for i=1 upto lua("MP.loaded('demo-data.lua')") :
        drawdot
            lua("MP.dot(",i,")") scaled cm
            withpen pencircle scaled 4mm
            withcolor darkred ;
    endfor ;
\stopMPcode
\stopbuffer

\typebuffer

Here we also omit the \type {decimal} because the \type {lua} macro is clever
enough to recognize it as a number.

\startlinecorrection[blank]
\getbuffer
\stoplinecorrection

By using some \METAPOST\ magic we can even go a step further in readability:

\startbuffer
\startMPcode{doublefun}
    lua.MP.load("demo-data.lua") ;

    for i=1 upto lua.MP.n() :
        drawdot
            lua.MP.dot(i) scaled cm
            withpen pencircle scaled 4mm
            withcolor darkred ;
    endfor ;

    for i=1 upto MP.n() :
        drawdot
            MP.dot(i) scaled cm
            withpen pencircle scaled 2mm
            withcolor white ;
    endfor ;
\stopMPcode
\stopbuffer

\typebuffer

Here we demonstrate that it also works ok in \type {double} mode, which makes
much sense when processing data from other sources. Watch how we omit the
type {lua.} prefix: the \type {MP} macro will deal with that.

\startlinecorrection[blank]
\getbuffer
\stoplinecorrection

So in the end we can simplify the code that we started with to:

\starttyping
\startMPcode{doublefun}
    for i=1 upto MP.loaded("demo-data.lua") :
        drawdot
            MP.dot(i) scaled cm
            withpen pencircle scaled 2mm
            withcolor darkred ;
    endfor ;
\stopMPcode
\stoptyping

\stopsection

% \startsection[title=Access to variables]
%
% The question with such mechanisms is always: how far should we go. Although
% \METAPOST\ is a macro language it has properties of procedural languages. It also
% has more introspective features at the user end. For instance, one can loop over
% the resulting picture and manipulate it. This means that we don't need full
% access to \METAPOST\ internals. However, it makes sense to provide access to
% basic variables: \type {numeric}, \type {string}, and \type {boolean}.
%
% \startbuffer
% draw textext(lua("mp.quoted('@0.15f',mp.get.numeric('pi')-math.pi)"))
%     ysized 1cm
%     withcolor darkred ;
% \stopbuffer
%
% \typebuffer
%
% In double mode you will get zero printed but in scaled mode we definitely get a
% difference:
%
% \startlinecorrection[blank]
% \processMPbuffer
% \stoplinecorrection
%
% \startbuffer
% boolean b ; b := true ;
% draw textext(lua("mp.quoted(mp.get.boolean('b') and 'yes' or 'no')"))
%     ysized 1cm
%     withcolor darkred ;
% \stopbuffer
%
% In the next example we use \type {mp.quoted} to make sure that indeed we pass a
% string. The \type {textext} macro can deal with numbers but an unquoted \type
% {yes} or \type {no} is asking for problems.
%
% \typebuffer
%
% Especially when more text is involved it makes sense to predefine a helper in
% the \type {MP} namespace if only because \METAPOST\ (currently) doesn't like
% newlines in the middle of a string, so a \type {lua} call has to be on one line.
%
% \startlinecorrection[blank]
% \processMPbuffer
% \stoplinecorrection
%
% Here is an example where \LUA\ does something that would be close to impossible,
% especially if more complex text is involved.
%
% % \enabletrackers[metapost.lua]
%
% \startbuffer
% string s ; s := "ΤΕΧ" ; % "τεχ"
% draw textext(lua("mp.quoted(characters.lower(mp.get.string('s')))"))
%     ysized 1cm
%     withcolor darkred ;
% \stopbuffer
%
% \typebuffer
%
% As you can see here, the whole repertoire of helper functions can be used in
% a \METAFUN\ definition.
%
% \startlinecorrection[blank]
% \processMPbuffer
% \stoplinecorrection
%
% \stopsection

\startsection[title=The library]

In \CONTEXT\ we have a dedicated runner, but for the record we mention the
low level constructor:

\starttyping
local m = mplib.new {
    ...
    script_runner = function(s) return loadstring(s)() end,
    script_error  = function(s) print(s) end,
    ...,
}
\stoptyping

An instance (in this case \type {m}) has a few extra methods. Instead you can use
the helpers in the library.

\starttabulate[|l|l|]
\HL
\NC \type {m:get_numeric(name)}       \NC returns a numeric (double) \NC \NR
\NC \type {m:get_boolean(name)}       \NC returns a boolean (\type {true} or \type {false}) \NC \NR
\NC \type {m:get_string (name)}       \NC returns a string \NC \NR
\HL
\NC \type {mplib.get_numeric(m,name)} \NC returns a numeric (double) \NC \NR
\NC \type {mplib.get_boolean(m,name)} \NC returns a boolean (\type {true} or \type {false}) \NC \NR
\NC \type {mplib.get_string (m,name)} \NC returns a string \NC \NR
\HL
\stoptabulate

In \CONTEXT\ the instances are hidden and wrapped in high level macros, so there
you cannot use these commands.

\stopsection

\startsection[title=\CONTEXT\ helpers]

The \type {mp} namespace provides the following helpers:

\starttabulate[|l|l|]
\HL
\NC \type {print(...)}                      \NC returns one or more values \NC \NR
\NC \type {fprint(fmt,...)}                 \NC returns a formatted result \NC \NR
\NC \type {boolean(b)}                      \NC returns \type {true} or \type {false} \NC \NR
\NC \type {numeric(f)}                      \NC returns a floating point number \NC \NR
\NC \type {integer(i)}                      \NC returns a whole number \NC \NR
\NC \type {points(i)}                       \NC returns a floating point with unit \type {pt} \NC \NR
\NC \type {pair(x,y)|(t)}                   \NC returns a proper pair \NC \NR
\NC \type {pairpoints(x,y)|(t)}             \NC returns a proper pair with unit \type {pt} \NC \NR
\NC \type {triplet(x,y,z)|(t)}              \NC returns a \RGB\ color \NC \NR
\NC \type {tripletpoints(x,y,z)|(t)}        \NC returns a \RGB\ color but with unit \type {pt} \NC \NR
\NC \type {quadruple(w,x,y,z)|(t)}          \NC returns a \CMYK\ color \NC \NR
\NC \type {quadruplepoints(w,x,y,z)|(t)}    \NC returns a \CMYK\ color but with unit \type {pt} \NC \NR
\NC \type {format(fmt,...)}                 \NC returns a formatted string \NC \NR
\NC \type {quoted(fmt,...)}
    \type {quoted(s)}                       \NC returns a (formatted) quoted string  \NC \NR
\NC \type {path(t[,connect][,close])}       \NC returns a connected (closed) path \NC \NR
\NC \type {pathpoints(t[,connect][,close])} \NC returns a connected (closed) path with units \type {pt} \NC \NR
\HL
\stoptabulate

The \type {mp.get} namespace provides the following helpers:

\starttabulate[|l|l|]
\HL
\NC \type {numeric(name)}      \NC gets a numeric from \METAPOST \NC \NR
\NC \type {boolean(name)}      \NC gets a boolean from \METAPOST \NC \NR
\NC \type {string(name)}       \NC gets a string from \METAPOST \NC \NR
\HL
\stoptabulate

\stopsection

\startsection[title=Paths]

In the meantime we got several questions on the \CONTEXT\ mailing list about turning
coordinates into paths. Now imagine that we have this dataset:

\startbuffer[dataset]
10 20 20 20 -- sample 1
30 40 40 60
50 10

10 10 20 30 % sample 2
30 50 40 50
50 20

10 20 20 10 # sample 3
30 40 40 20
50 10
\stopbuffer

\typebuffer[dataset]

In this case I've put the data in a buffer so that it can be shown
here as well as used in a demo. Watch how we can add comments. The
following code converts this into a table with three subtables.

\startbuffer
\startluacode
  MP.myset = mp.dataset(buffers.getcontent("dataset"))
\stopluacode
\stopbuffer

\typebuffer \getbuffer

We use the \type {MP} (user) namespace to store the table. Next we turn
these subtables into paths:

\startbuffer
\startMPcode
  for i=1 upto lua("mp.print(mp.n(MP.myset))") :
    draw
      lua("mp.path(MP.myset[" & decimal i & "])")
      xysized (HSize-.25ExHeight,10ExHeight)
      withpen pencircle scaled .25ExHeight
      withcolor basiccolors[i]/2 ;
  endfor ;
\stopMPcode
\stopbuffer

\typebuffer

This gives:

\startlinecorrection[blank] \getbuffer \stoplinecorrection

Instead we can fill the path in which case we also need to close it. The
\type {true} argument deals with that:

\startbuffer
\startMPcode
  for i=1 upto lua("mp.print(mp.n(MP.myset))") :
    path p ; p :=
      lua("mp.path(MP.myset[" & decimal i & "],true)")
      xysized (HSize,10ExHeight) ;
    fill p
      withcolor basiccolors[i]/2
      withtransparency (1,.5) ;
  endfor ;
\stopMPcode
\stopbuffer

\typebuffer

We get:

\startlinecorrection[blank] \getbuffer \stoplinecorrection

\startbuffer
\startMPcode
  for i=1 upto lua("mp.print(mp.n(MP.myset))") :
    path p ; p :=
      lua("mp.path(MP.myset[" & decimal i & "])")
      xysized (HSize,10ExHeight) ;
    p :=
      (xpart llcorner boundingbox p,0) --
      p --
      (xpart lrcorner boundingbox p,0) --
      cycle ;
    fill p
      withcolor basiccolors[i]/2
      withtransparency (1,.25) ;
  endfor ;
\stopMPcode
\stopbuffer

The following makes more sense:

\typebuffer

So this gives:

\startlinecorrection[blank] \getbuffer \stoplinecorrection

This (area) fill is so common that we have a helper for it:

\startbuffer
\startMPcode
  for i=1 upto lua("mp.size(MP.myset)") :
    fill area
      lua("mp.path(MP.myset[" & decimal i & "])")
      xysized (HSize,5ExHeight)
      withcolor basiccolors[i]/2
      withtransparency (2,.25) ;
  endfor ;
\stopMPcode
\stopbuffer

\typebuffer

So this gives:

\startlinecorrection[blank] \getbuffer \stoplinecorrection

% A variant call is the following: \footnote {Getting that to work properly in the
% library was non||trivial as the loop variable \type {i} is an abstract nameless
% variable at the \METAPOST\ end. When investigating this Luigi Scarso and I found out
% that the internals of \METAPOST\ are not really geared for interfacing this way
% but in the end it worked out well.}
%
% \startbuffer
% \startMPcode
%   for i=1 upto lua("mp.size(MP.myset)") :
%     fill area
%       lua("mp.path(MP.myset[mp.get.numeric('i')])")
%       xysized (HSize,5ExHeight)
%       withcolor basiccolors[i]/2
%       withtransparency (2,.25) ;
%   endfor ;
% \stopMPcode
% \stopbuffer
%
% \typebuffer
%
% The result is the same:
%
% \startlinecorrection[blank] \getbuffer \stoplinecorrection
%
% \startbuffer
% \startluacode
%   MP.mypath = function(i)
%     return mp.path(MP.myset[mp.get.numeric(i)])
%   end
% \stopluacode
% \stopbuffer
%
% \typebuffer \getbuffer
%
% \startbuffer
% \startMPcode
%   for i=1 upto lua("mp.size(MP.myset)") :
%     fill area
%       lua("MP.mypath('i')")
%       xysized (HSize,5ExHeight)
%       withcolor basiccolors[i]/2
%       withtransparency (2,.25) ;
%   endfor ;
% \stopMPcode
% \stopbuffer
%
% \typebuffer

This snippet of \METAPOST\ code still looks kind of horrible so how can we make
it look better? Here is an attempt, First we define a bit more \LUA:

\startbuffer
\startluacode
local data = mp.dataset(buffers.getcontent("dataset"))

MP.dataset = {
  Line = function(n) mp.path(data[n]) end,
  Size = function()  mp.size(data)    end,
}
\stopluacode
\stopbuffer

\typebuffer \getbuffer

We can now make the \METAPOST\ look more natural. Of course this is possible
because in \METAFUN\ the \type {lua} macro does some extra work.

\startbuffer
\startMPcode
  for i=1 upto lua.MP.dataset.Size() :
    path p ; p :=
      lua.MP.dataset.Line(i)
      xysized (HSize-ExHeight,20ExHeight) ;
    draw
      p
      withpen pencircle scaled .25ExHeight
      withcolor basiccolors[i]/2 ;
    drawpoints
      p
      withpen pencircle scaled ExHeight
      withcolor basiccolors[i]/2 ;
  endfor ;
\stopMPcode
\stopbuffer

\typebuffer

As expected, we get the desired result:

\startlinecorrection[blank] \getbuffer \stoplinecorrection

Once we start making things look nicer and more convenient, we quickly end up
with helpers like the once in the next example. First we save some demo data
in files:

\startbuffer
\startluacode
  io.savedata("foo.tmp","10 20 20 20 30 40 40 60 50 10")
  io.savedata("bar.tmp","10 10 20 30 30 50 40 50 50 20")
\stopluacode
\stopbuffer

\typebuffer \getbuffer

We load the data in datasets:

\startbuffer
\startMPcode
  lua.mp.datasets("load","foo","foo.tmp") ;
  lua.mp.datasets("load","bar","bar.tmp") ;
  fill area
    lua.mp.datasets("foo","line")
    xysized (HSize/2-EmWidth-.25ExHeight,10ExHeight)
    withpen pencircle scaled .25ExHeight
    withcolor darkyellow ;
  fill area
    lua.mp.datasets("bar","line")
    xysized (HSize/2-EmWidth-.25ExHeight,10ExHeight)
    shifted (HSize/2+EmWidth,0)
    withpen pencircle scaled .25ExHeight
    withcolor darkred ;
\stopMPcode
\stopbuffer

\typebuffer

Because the datasets are stored by name we can use them without worrying about
them being forgotten:

\startlinecorrection[blank] \getbuffer \stoplinecorrection

If no tag is given, the filename (without suffix) is used as tag, so the following is
valid:

\starttyping
\startMPcode
  lua.mp.datasets("load","foo.tmp") ;
  lua.mp.datasets("load","bar.tmp") ;
\stopMPcode
\stoptyping

The following methods are defined for a dataset:

\starttabulate[|l|pl|]
\HL
\NC \type {method} \NC usage \NC \NR
\HL
\NC \type {size}   \NC the number of subsets in a dataset \NC \NR
\NC \type {line}   \NC the joined pairs in a dataset making a non|-|closed path \NC \NR
\NC \type {data}   \NC the table containing the data (in subsets, so there is always at least one subset) \NC \NR
\HL
\stoptabulate

{\em In order avoid interference with suffix handling in \METAPOST\ the methods
start with an uppercase character.}

\stopsection

\startsection[title=Passing variables]

You can pass variables from \METAPOST\ to \CONTEXT. Originally that happened via
a temporary file and so called \METAPOST\ specials. Nowadays it's done via \LUA.
Here is an example:

\startbuffer
\startMPcalculation

passvariable("version","1.0") ;
passvariable("number",123) ;
passvariable("string","whatever") ;
passvariable("point",(1.5,2.8)) ;
passvariable("triplet",(1/1,1/2,1/3)) ;
passvariable("quad",(1.1,2.2,3.3,4.4)) ;
passvariable("boolean",false) ;
passvariable("path",fullcircle scaled 1cm) ;
save p ; path p[] ; p[1] := fullcircle ; p[2] := fullsquare ;
passarrayvariable("list",p,1,2,1) ; % first last step
\stopMPcalculation
\stopbuffer

\typebuffer

\getbuffer

We can visualize the result with

\startbuffer
\startluacode
context.tocontext(metapost.variables)
\stopluacode
\stopbuffer

\typebuffer

\getbuffer

In \TEX\ you can access these variables as follows:

\startbuffer
\MPrunvar{version}
\MPruntab{quad}{3}
(\MPrunset{triplet}{,})

$(x,y) = (\MPruntab{point}{1},\MPruntab{point}{2})$
$(x,y) = (\MPrunset{point}{,})$
\stopbuffer

\typebuffer

This becomes: % we need a hack as we cross pages and variables get replace then

\startlines
\getbuffer
\stoplines

Here we passed the code between \type {\startMPcalculation} and \type
{\stopMPcalculation} which does not produce a graphic and therefore takes no
space in the flow. Of course it also works with normal graphics.

\startbuffer
\startMPcode
path p ; p := fullcircle xyscaled (10cm,2cm) ;
path b ; b := boundingbox p ;
startpassingvariable("mypath")
    passvariable("points",p) ;
    startpassingvariable("metadata")
        passvariable("boundingbox",boundingbox p) ;
    stoppassingvariable ;
stoppassingvariable ;
fill p withcolor .625red ;
draw b withcolor .625yellow ;
\stopMPcode
\stopbuffer

\typebuffer

\startlinecorrection[blank]
    \getbuffer
\stoplinecorrection

This time we get:

\ctxlua{context.tocontext(metapost.variables)}

You need to be aware of the fact that a next graphic resets the previous
variables. You can easily overcome that limitation by saving the variables (in
\LUA). It helps that when a page is being shipped out (which can involve
graphics) the variables are protected. You can push and pop variable sets with
\type {\MPpushvariables} and \type {\MPpopvariables}. Because you can nest
the \type {start}||\type{stop} pairs you can create quite complex indexed
and hashed tables. If the results are not what you expect, you can enable a
tracker to follow what gets passed:

\starttyping
\enabletrackers[metapost.variables]
\stoptyping

Serializing variables can be done with the \type {tostring} macro, for instance:

\startbuffer
\startMPcode
message("doing circle",fullcircle);
draw fullcircle ;
\stopMPcode
\stopbuffer

In this case the \type {tostring} is redundant as the message already does the
serialization.

\stopsection

\startsection[title={Interference}]

In this section we will discuss a potential conflict with other mechanisms,
especially primitives and macros. A simple example of using the interface is:

\startbuffer
\startluacode
function MP.AnExample(str)
    mp.aux.quoted(string.reverse(str))
end
\stopluacode

\startMPcode
draw textext(lua.MP.AnExample("Hi there!"))
    rotated 45
    ysized 2cm ;
\stopMPcode
\stopbuffer

\typebuffer

\startlinecorrection
\getbuffer
\stoplinecorrection

The \type {mp} namespace is reserved for functionality provided by \CONTEXT\
itself so you should not polute it with your own code. Instead use the \type {MP}
namespace and mix in some uppercase characters.

Here you see a subnamespace \type {aux} which is where officially the helpers are
organized but they are also accessible directly (as shown in previous sections).
The reason for the \type {aux} namespace is, apart from propection aginst
redefinition, also that we have \type {get} and \type {set} namespaces and there
might be more in the future. At the \LUA\ end you can best use these namespaces
because they are less likely to be accidentally overwritten by user code.

As mentioned, there can still be conflicts. For instance the following will not
work:

\startbuffer
\startluacode
function MP.reverse(str)
    mp.aux.quoted(string.reverse(str))
end
\stopluacode

\startMPcode
draw textext(lua.MP.reverse("Hi there!"))
    rotated -45
    ysized 2cm ;
\stopMPcode
\stopbuffer

\typebuffer

% \startlinecorrection
% \getbuffer
% \stoplinecorrection

The reason is that \type {reverse} gets expanded as part of parsing the macro
name and this command expects an expression. A way out of this is the following:

\startbuffer
\startluacode
function MP.reverse(str)
    mp.aux.quoted(string.reverse(str))
end
\stopluacode

\startMPcode
draw textext(lua.MP("reverse","Hi there!"))
    rotated -45
    ysized 2cm ;
\stopMPcode
\stopbuffer

\typebuffer

\startlinecorrection
\getbuffer
\stoplinecorrection

You can add litle bit of protection for your own code by using a prefix in the
name, like:

\startbuffer
\startluacode
MP["mynamespace.reverse"] = function(str)
    mp.aux.quoted(string.reverse(str))
end
\stopluacode

\startMPcode
draw textext(lua.MP("mynamespace.reverse","Hi there!"))
    rotated -90
    ysized 2cm ;
\stopMPcode
\stopbuffer

\typebuffer

\startlinecorrection
\getbuffer
\stoplinecorrection

\stopsection

\startsection[title=Predefined properties]

As mentioned, the \type {mp} namespace is reserved for commands that come
with the \CONTEXT|-|\METAFUN\ combination. For instance, a lot of layout
related calls are there::

\startcolumns[n=3]
\starttyping
BackSpace
BaseLineSkip
BodyFontSize
BottomDistance
BottomHeight
BottomSpace
CurrentColumn
CurrentHeight
CurrentWidth
CutSpace
EmWidth
ExHeight
FooterDistance
FooterHeight
HeaderDistance
HeaderHeight
InnerEdgeDistance
InnerEdgeWidth
InnerMarginDistance
InnerMarginWidth
LastPageNumber
LayoutColumnDistance
LayoutColumns
LayoutColumnWidth
LeftEdgeDistance
LeftEdgeWidth
LeftMarginDistance
LeftMarginWidth
LineHeight
MakeupHeight
MakeupWidth
NOfColumns
NOfPages
NOfPages
NOfSubPages
OuterEdgeDistance
OuterEdgeWidth
OuterMarginDistance
OuterMarginWidth
PageDepth
PageFraction
PageNumber
PageNumber
PageOffset
PaperBleed
PaperHeight
PaperWidth
PrintPaperHeight
PrintPaperWidth
RealPageNumber
RealPageNumber
RightEdgeDistance
RightEdgeWidth
RightMarginDistance
RightMarginWidth
SpineWidth
StrutDepth
StrutHeight
SubPageNumber
TextHeight
TextWidth
TopDistance
TopHeight
TopSkip
TopSpace
\stoptyping
\stopcolumns

There all return dimensions, contrary to the next few that return a boolean:

\startcolumns[n=3]
\starttyping
OnRightPage
OnOddPage
InPageBody
\stoptyping
\stopcolumns

There are also calls related to backgrounds:

\startcolumns[n=3]
\starttyping
OverlayWidth
OverlayHeight
OverlayDepth
OverlayLineWidth
OverlayOffset
\stoptyping
\stopcolumns

And one related to color:

\startcolumns[n=3]
\starttyping
NamedColor
\stoptyping
\stopcolumns

In most cases such \type {lua.mp.command()} calls have a \METAPOST\ macro with
the same name defined.

\stopsection

\startsection[title=Accessing paths]

A path in \METAPOST\ is internally a linked list of knots and each knot has a
coordinate and two control points. Access to specific points of very large path
can be somewhat slow. First of all, the lookup start at the beginning and when
you use fractions (say halfway between point 4 and 5) the engine has to find the
spot. For this (and other reasons not mentioned here) we have a way to access
paths different, using \LUA\ behind the scenes.

\startbuffer
path p ; p := fullcircle xysized (4cm,2cm) ;
for i inpath p:
    drawdot  leftof i withpen pencircle scaled 2mm withcolor darkred ;
    drawdot pointof i withpen pencircle scaled 3mm withcolor darkgreen ;
    drawdot rightof i withpen pencircle scaled 2mm withcolor darkblue ;
endfor ;
draw for i inpath p:
    pointof i .. controls (leftof i) and (rightof i) ..
endfor cycle withpen pencircle scaled .5mm withcolor white ;

p := p shifted (5cm,0) ;
draw for i inpath p:
    pointof i --
endfor cycle withpen pencircle scaled .5mm withcolor .5white ;
for i inpath p:
    drawdot pointof i withpen pencircle scaled 3mm withcolor darkgreen ;
endfor ;
\stopbuffer

\typebuffer

Here we access the main coordinate and the two control points. The last draw is of course
just mimicking drawing the path.

\startlinecorrection[blank]
\processMPbuffer
\stoplinecorrection

\stopsection

\startsection[title=Acessing \TEX]

In \MKIV\ and \LMTX\ it is possible to access \TEX\ registers and macros from the
\METAPOST\ end. Let's first define and set some:

\startbuffer
\newdimen\MyMetaDimen  \MyMetaDimen = 2mm
\newcount\MyMetaCount  \MyMetaCount = 10
\newtoks \MyMetaToks   \MyMetaToks  = {\bfd \TeX}
     \def\MyMetaMacro                 {not done}
\stopbuffer

\typebuffer \getbuffer

\startbuffer
\startMPcode
    for i=1 upto getcount("MyMetaCount") :
        draw fullcircle scaled (i * getdimen("MyMetaDimen")) ;
    endfor ;
    draw textext(gettoks("MyMetaToks")) xsized 15mm withcolor darkred ;
    setglobaldimen("MyMetaDimen", bbwidth(currentpicture)) ;
    setglobalmacro("MyMetaMacro", "done") ;
\stopMPcode
\stopbuffer

\typebuffer

\startlinecorrection[blank]
    \getbuffer
\stoplinecorrection

We can now look at the two updated globals where \type {\MyMetaMacro: \the\MyMetaDimen}
typesets: {\tttf \MyMetaMacro: \the\MyMetaDimen}. As demonstrated you can best define your
own registers but in principle you can also access system ones, like \type {\scratchdimen}
and friends.

\stopsection

\startsection[title=Abstraction]

We will now stepwise implement some simple helpers for accessing data in files.
The examples are kind of useless but demonstrate how interfaces evolved. The
basic command to communicate with \LUA\ is \type {runscript}. In this example
we will load a (huge) file and run over the lines.

\starttyping
\startMPcode{doublefun}
  save q ; string q ; q := "'\\" & ditto & "'" ;
  runscript (
    "GlobalData = string.splitlines(io.loaddata('foo.tmp')) return ''"
  ) ;
  numeric l ; l = runscript (
    "return string.format('\letterpercent q',\letterhash GlobalData)"
  );
  for i=1 step 1 until l :
    l := length ( runscript (
      "return string.format('\letterpercent q',GlobalData[" & decimal i & "])"
    ) ) ;
  endfor ;
  draw textext(decimal l);
\stopMPcode
\stoptyping

The \type {runscript} primitive takes a string and should return a string (in
\LUAMETATEX\ you can also return nothing). This low level solution will serve as
our benchmark: it takes 2.04 seconds on the rather large (64MB) test file with
10.000 lines.

The code looks somewhat clumsy. This is because in \METAPOST\ escaping is not
built in so one has to append a double quote character using \type {char 34} and
the \type {ditto} string is defined as such. This mess is why in \CONTEXT\ we
have an interface:

\starttyping
\startMPcode{doublefun}
  lua("GlobalData = string.splitlines(io.loaddata('foo.tmp'))") ;
  numeric l ;
  for i=1 step 1 until lua("mp.print(\#GlobalData)") :
    l := length(lua("mp.quoted(GlobalData[" & decimal i & "])")) ;
  endfor ;
  draw textext(decimal l);
\stopMPcode
\stoptyping

As expected we pay a price for the additional overhead, so this time we need 2.28
seconds to process the file. The return value of a run is a string that is fed
into \type {scantokens}. Here \type {print} function prints the number as string
and that gets scanned back to a number. The \type {quoted} function returns a
string in a string so when we're back in \METAPOST\ that gets scanned as string.

When code is used more frequently, we can make a small library, like this:

\starttyping
\startluacode
  local MyData = { }
  function mp.LoadMyData(filename)
    MyData = string.splitlines(io.loaddata(filename))
  end
  local mpprint  = mp.print
  local mpquoted = mp.quoted
  function mp.MyDataSize()
    mpprint(#MyData)
  end
  function mp.MyDataString(i)
    mpquoted(MyData[i] or "")
  end
\stopluacode
\stoptyping

It is not that hard to imagine a more advanced mechanisms where data from multiple
files can be handled at the same time. This code is used as:

\starttyping
\startMPcode{doublefun}
  lua.mp.LoadMyData("foo.tmp") ;
  numeric l ;
  for i=1 step 1 until lua.mp.MyDataSize() :
    l := length(lua.mp.MyDataString(i)) ;
  endfor ;
  draw textext(decimal l);
\stopMPcode
\stoptyping

The \type {mp} namespace at the \LUA\ end is a subnamespace at the \METAPOST\
end. This solution needs 2.20 seconds so we're still slower than the first one,
but in \LUAMETATEX\ with \LMTX we can do better. First the \LUA\ code:

\starttyping
\startluacode
  local injectnumeric = mp.inject.numeric
  local injectstring  = mp.inject.string
  local MyData = { }
  function mp.LoadMyData(filename)
    MyData = string.splitlines(io.loaddata(filename))
  end
  function mp.MyDataSize()
    injectnumeric(#MyData)
  end
  function mp.MyDataString(i)
    injectstring(MyData[i] or "")
  end
\stopluacode
\stoptyping

This time we use injectors. The mentioned \type {print} helpers serialize data so
numbers, pairs, colors etc are converted to a string that represents them that is
fed back to \METAPOST\ after the snippet is run. Multiple prints are collected
into one string. An injecter follows a more direct route: it pushes back a proper
\METAPOST\ data type.

\starttyping
\startMPcode{doublefun}
  lua.mp.LoadMyData("foo.tmp") ;
  numeric l ;
  for i=1 step 1 until lua.mp.MyDataSize() :
    l := length(lua.mp.MyDataString(i)) ;
  endfor ;
  draw textext(decimal l);
\stopMPcode
\stoptyping

This usage brings us down to 1.14 seconds, so we're still not good. The next
variant is performing similar: 1.05 seconds.

\starttyping
\startMPcode{doublefun}
  runscript("mp.LoadMyData('foo.tmp')") ;
  numeric l ;
  for i=1 step 1 until runscript("mp.MyDataSize()") :
    l := length(runscript("mp.MyDataString(" & decimal i & ")")) ;
  endfor ;
  draw textext(decimal l);
\stopMPcode
\stoptyping

We will now delegate scanning to the \LUA\ end.

\starttyping
\startluacode
  local injectnumeric = mp.inject.numeric
  local injectstring  = mp.inject.string
  local scannumeric   = mp.scan.numeric
  local scanstring    = mp.scan.string
  local MyData = { }
  function mp.LoadMyData()
    MyData = string.splitlines(io.loaddata(scanstring()))
  end
  function mp.MyDataSize()
    injectnumeric(#MyData)
  end
  function mp.MyDataString()
    injectstring(MyData[scannumeric()] or "")
  end
\stopluacode
\stoptyping

This time we are faster than the clumsy code we started with: 0.87 seconds.

\starttyping
\startMPcode{doublefun}
  runscript("mp.LoadMyData()") "foo.tmp" ;
  numeric l ;
  for i=1 step 1 until runscript("mp.MyDataSize()") :
    l := length(runscript("mp.MyDataString()") i) ;
  endfor ;
  draw textext(decimal l);
\stopMPcode
\stoptyping

In \LMTX\ we can add some more abstraction. Performance is about the same and
sometimes a bit faster but that depends on extreme usage: you need thousands of
call to notice.

\starttyping
\startluacode
  local injectnumeric = mp.inject.numeric
  local injectstring  = mp.inject.string
  local scannumeric   = mp.scan.numeric
  local scanstring    = mp.scan.string
  local MyData = { }
  metapost.registerscript("LoadMyData", function()
    MyData = string.splitlines(io.loaddata(scanstring()))
  end)
  metapost.registerscript("MyDataSize", function()
    injectnumeric(#MyData)
  end)
  metapost.registerscript("MyDataString", function()
    injectstring(MyData[scannumeric()] or "")
  end)
\stopluacode
\stoptyping

We have the same scripts but we register them. At the \METAPOST\ end we resolve
the registered scripts and then call \type {runscript} with the (abstract) numeric
value:

\starttyping
\startMPcode{doublefun}
  newscriptindex my_script_LoadMyData   ;
  newscriptindex my_script_MyDataSize   ;
  newscriptindex my_script_MyDataString ;

  my_script_LoadMyData   := scriptindex "LoadMyData"   ;
  my_script_MyDataSize   := scriptindex "MyDataSize"   ;
  my_script_MyDataString := scriptindex "MyDataString" ;

  runscript my_script_LoadMyData "foo.tmp" ;
  numeric l ;
  for i=1 step 1 until runscript my_script_MyDataSize :
    l := length(my_script_MyDataString i) ;
  endfor ;
  draw textext(decimal l);
\stopMPcode
\stoptyping

This is of course nicer:

\starttyping
\startMPcode{doublefun}
  def LoadMyData  (expr s) = runscript my_script_LoadMyData   s enddef ;
  def MyDataSize           = runscript my_script_MyDataSize     enddef ;
  def MyDataString(expr i) = runscript my_script_MyDataString i enddef ;

  LoadMyData("foo.tmp") ;
  numeric l ;
  for i=1 step 1 until MyDataSize :
    l := length(MyDataString(i)) ;
  endfor ;
  draw textext(decimal l);
\stopMPcode
\stoptyping

So, to sumarize, there are many ways to look at this: verbose direct ones
but also nicely abstract ones.

\stopsection

% The plugins (like those dealing with text) also use calls in the \type {mp}
% namespace but they have sort of protected names, starting with \type {mf_}. These
% are visible but not meant to be used by users. Not only can their name change,
% their functionality can as well.
%
% The following are actually private as they have related macros but also have a
% public alias:
%
% \startlines
% lua.mp.pathlength(name)
% lua.mp.pathpoint(i)
% lua.mp.pathleft(i)
% lua.mp.pathright(i)
% lua.mp.pathreset()
% \stoplines
%
% They are meant for special high|-|performance path access, for example:
%
% \startlinecorrection
% \startMPcode
%     save p, q, r;
%     path p ; p := for i=1 upto 1000 :
%         (i,0) -- (i,4 + uniformdeviate 4) --
%     endfor cycle ;
%     fill p xysized (TextWidth,20mm) withcolor red ;
%
%     path q ; q := for i=1 upto lua.mp.pathlength("p") :
%         if (i mod 4) == 0 : lua.mp.pathpoint(i) -- fi
%     endfor cycle ;
%     fill q xysized (TextWidth,2cm) shifted (0,-45mm) withcolor green ;
%
%     path r ; r := for i inpath p :
%         if not odd (i) : pointof i -- fi
%     endfor cycle ;
%     fill r xysized (TextWidth,2cm) shifted (0,-70mm) withcolor blue ;
% \stopMPcode
% \stoplinecorrection
%
% Because a lookup of a point in \METAPOST\ is a linear lookup over a linked list
% for very large paths the gain is significant when using \LUA\ because there the
% points are stored in an indexed table. The left and right points can be used
% vebose like
%
% \starttyping
% lua.mp.pathpoint(i) controls lua.mp.pathleft(i) and lua.mp.pathright(i)
% \stoptyping
%
% or more terse:
%
% \starttyping
% pointof i controls leftof i and rightof i
% \stoptyping
%
% Beware: this kind of trickery is {\em only} needed when you have very large paths
% that are to be manipulated and the. Otherwise it's overkill.
%
% \stopsection
%
% \startsection[title={Interfacing to \TEX}]
%
% The next bunch of calls is for accessing \TEX\ registers. You can set their
% values and get them as well.
%
% \starttyping
% lua.mp.getmacro(k)
% lua.mp.getdimen(k)
% lua.mp.getcount(k)
% lua.mp.gettoks (k)
%
% lua.mp.setmacro(k,v)
% lua.mp.setdimen(k,v)
% lua.mp.setcount(k,v)
% lua.mp.settoks (k,v)
% \stoptyping
%
% When you mess around with variables and run into issues it might help to
% report values on the console. The \type {report} function does this:
%
% \starttyping
% lua.mp.report(a,b)
% \stoptyping
%
% \startlinecorrection
% \startMPcode
% lua.mp.report("status","a circle") ;
% fill fullcircle xyscaled (2cm,1cm) withcolor red ;
% lua.mp.report("status","its boundingbox [@N,@N,@N,@N]",
%     xpart llcorner currentpicture, ypart llcorner currentpicture,
%     xpart urcorner currentpicture, ypart urcorner currentpicture
% ) ;
% draw boundingbox currentpicture withcolor blue ;
% report("status","the size: @Nbp x @Nbp",
%     bbwidth(currentpicture), bbheight(currentpicture)
% ) ;
% message("done") ;
% \stopMPcode
% \stoplinecorrection
%
% The console shows:
%
% \starttyping
% metapost > status : a circle
% metapost > status : its boundingbox [-28.34645,-14.17323,28.34645,14.17323]
% metapost > status : the size: 57.1929bp x 28.84647bp
% metapost > message : done
% \stoptyping
%
% There are two more getters. These can be used to access the specific graphic
% related variables set at the \TEX\ end.
%
% \starttyping
% mp.texvar(name)
% mp.texstr(name)
% \stoptyping
%
% If you have adaptive styles you might want to test for modes.
%
% \starttyping
% if lua.mp.mode("screen") : % or processingmode
%     % some action only needed for screen documents
% fi ;
% if lua.mp.systemmode("first") :
%     % some action for the first run
% fi ;
% \stoptyping
%
% For convenience these are wrapped into macros:
%
% \starttyping
% if texmode("screen") :
%     % some action only needed for screen documents
% fi ;
% if systemmode("first") :
%     % some action for the first run
% fi ;
% \stoptyping
%
% When you implement your own helpers you can fall back on some auxiliary functions
% in the \type {mp} namespace. Actually these are collected in \type {mp.aux} and
% thereby protected from being overwritten by mistake. Here they are:
%
% % mp.flush()
% % mp.size(t)
%
% \stopsection

% \startsection[title={Interfacing to \METAPOST}]
%
% There is also experimental access to some of the \METAPOST\ internals. In order
% to deal with (large) paths a few more iterator related helpers are provided too.
%
% % mp.set (numeric string path boolean)
%
% \starttyping
% n = mp.getnumeric(name)
% s = mp.getstring(name)
% b = mp.getboolean(name)
% p = mp.getpath(name)
% \stoptyping
%
% Although it might look like \METAPOST\ supports arrays the reality is that it
% doesn't really. There is a concept of suffixes but internally these are just a
% way to compose macros. The following test is one that is used in one of the
% \METAFUN\ modules written by Alan Braslau.
%
% \startlinecorrection
% \startMPcode
% path p, q[] ;
%
% if lua.mp.isarray(str q[1]) :
%     fill fullcircle scaled 1cm withcolor red ;
%     draw textext(lua.mp.prefix(str q[1])) withcolor white;
% else :
%     fill fullsquare scaled 1cm withcolor blue ;
% fi ;
%
% currentpicture := currentpicture shifted (-2cm,0) ;
%
% if lua.mp.isarray(str p) :
%     fill fullcircle scaled 1cm withcolor red ;
% else :
%     fill fullsquare scaled 1cm withcolor blue ;
% fi ;
% \stopMPcode
% \stoplinecorrection
%
% Another helper relates to extensions that \METAFUN\ adds to \METAPOST. When you
% iterate over a picture you can recognize these as objects. The next code shows
% the \LUA\ call as well as the more convenient macro call. So, \type {textext}
% clearly is a foreign object.
%
% \startlinecorrection
% \startMPcode
% picture p ; p := image (
%     fill fullcircle scaled 1cm withcolor red ;
%     draw textext("ok") withcolor white ;
% ) ;
%
% for i within p :
%     if not picture i :
%         draw i ysized 4cm ;
%     elseif isobject i :
%         draw i xsized 3cm ;
%     else :
%         draw i ysized 4cm ;
%     fi ;
% endfor ;
%
% currentpicture := currentpicture shifted (-6cm,0) ;
%
% for i within p :
%     if isobject(i) :
%         draw i xsized 5cm ;
%     else :
%         draw i ysized 3cm withcolor blue;
%     fi ;
% endfor ;
% \stopMPcode
% \stoplinecorrection

% not yet, still experimental:
%
% mp.dataset(str)
% mp.n(t)

% not for users:
%
% mp.defaultcolormodel()

% rather specialized:
%
% mp.positionpath(name)
% mp.positioncurve(name)
% mp.positionbox(name)
% mp.positionxy(name)
% mp.positionpage(name)
% mp.positionregion(name)
% mp.positionwhd(name)
% mp.positionpxy(name)
% mp.positionanchor()


% mp.cleaned
% mp.format(fmt,str)
% mp.formatted(fmt,...)
% mp.graphformat(fmt,num)

\stopsection

\stopchapter

% \startMPcode{doublefun}
%     numeric n ; n := 123.456 ;
%     lua("print('>>>>>>>>>>>> number',mp.get.number('n'))") ;
%     lua("print('>>>>>>>>>>>> number',mp.get.boolean('n'))") ;
%     lua("print('>>>>>>>>>>>> number',mp.get.string('n'))") ;
%     boolean b ; b := true ;
%     lua("print('>>>>>>>>>>>> boolean',mp.get.number('b'))") ;
%     lua("print('>>>>>>>>>>>> boolean',mp.get.boolean('b'))") ;
%     lua("print('>>>>>>>>>>>> boolean',mp.get.string('b'))") ;
%     string s ; s := "TEST" ;
%     lua("print('>>>>>>>>>>>> string',mp.get.number('s'))") ;
%     lua("print('>>>>>>>>>>>> string',mp.get.boolean('s'))") ;
%     lua("print('>>>>>>>>>>>> string',mp.get.string('s'))") ;
% \stopMPcode

% \usemodule[graph]
%
% \startluacode
%     local d = nil
%     function MP.set(data)
%         d = data
%     end
%     function MP.n()
%         mp.print(d and #d or 0)
%     end
%     function MP.get(i,j)
%         mp.print(d and d[i] and d[i][j] or 0)
%     end
% \stopluacode
%
% \startluacode
%     MP.set {
%         { 1, 0.5, 2.5 },
%         { 2, 1.0, 3.5 },
%     }
% \stopluacode
%
% \startMPpage[instance=graph,offset=2mm]
%
% draw begingraph(3cm,5cm);
%     numeric a[];
%     for j = 1 upto MP.n() :
%         path b;
%         augment.b(MP.get(j,1),MP.get(j,2));
%         augment.b(MP.get(j,1),MP.get(j,3));
%         setrange(0,0,3,4);
%         gdraw b;
%         endfor ;
% endgraph ;
% \stopMPpage

% \starttext
%
% % \enabletrackers[metapost.variables]
%
% \startMPcode
%     numeric n[]   ; for i=1 upto 10: n[i] := 1/i ; endfor ;
%     path    p[]   ; for i=1 upto 10: p[i] := fullcircle xyscaled (cm*i,cm/i) ; endfor ;
%     numeric r[][] ; for i=1 upto 4 : for j=1 upto 3 : r[i][j] := uniformdeviate(1) ; endfor ; endfor ;
%     pair    u[][] ; for i=1 step 0.5 until 4 : for j=1 step 0.1 until 2 : u[i][j] := (i,j) ; endfor ; endfor ;
%
%     passvariable("x",12345) ;
%     passarrayvariable("n-array",n,1,7,1) ;
%     passarrayvariable("p-array",p,1,7,1) ;
%     passvariable("p",(1,1) .. (2,2)) ;
%
%     startpassingvariable("b")
%         for i=1 upto 4 :
%             startpassingvariable(i)
%                 for j=1 upto 3 :
%                     passvariable(j,r[i][j])
%                 endfor
%             stoppassingvariable
%         endfor
%     stoppassingvariable ;
%
%     startpassingvariable("a")
%         startpassingvariable("test 1")
%             passvariable(1,123)
%             passvariable(2,456)
%         stoppassingvariable ;
%         startpassingvariable("test 2")
%             passvariable(0,123)
%             passvariable(1,456)
%             passvariable(2,789)
%             passvariable(999,987)
%         stoppassingvariable ;
%         startpassingvariable("test 3")
%             passvariable("first",789)
%             passvariable("second",987)
%         stoppassingvariable
%     stoppassingvariable ;
%
%     startpassingvariable("c")
%         for i=1 step 0.5 until 4 :
%             startpassingvariable(i)
%                 for j=1 step 0.1 until 2 :
%                     passvariable(j,u[i][j])
%                 endfor
%             stoppassingvariable
%         endfor
%     stoppassingvariable ;
%
%     draw fullcircle scaled 1cm ;
% \stopMPcode
%
% \ctxluacode{inspect(metapost.variables)}
%
% \ctxcommand{mprunvar("x")}

\stopcomponent
