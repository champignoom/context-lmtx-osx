\startcomponent ma-cb-en-structure

\enablemode[**en-us]

\project ma-cb

\startchapter[title=Defining a document]

Every document is started with \type{\starttext} and closed with
\type{\stoptext}. All textual input is placed between these two commands and
\CONTEXT\ will only process that information.

Setup information is placed in the set up area just before \type{\starttext}.

\setuptyping
    [escape=yes]

\definestartstop
    [comment][style={\rm}]

\starttyping

\setupbodyfont[12pt]              /BTEX \comment{setuparea of document} /ETEX

\starttext
This is a one line document.      /BTEX \comment{your text} /ETEX
\stoptext
\stoptyping

The definition of a (very simple) book could look something like this:

\startbuffer
\starttext

\startstandardmakeup
  \midaligned{From Hasselt to America}
  \midaligned{by}
  \midaligned{J. Jonker and C. van Marle}
\stopstandardmakeup

\title{Foreword}

\chapter{Introduction}

\chapter{The Rensselaer family}

\chapter{The Lansing family}

\chapter{The Cuyler family}

\chapter{Appendix: Photos}

\stoptext
\stopbuffer

\typebuffer

\CONTEXT\ comes with a predefined overall structure in which the document is
divided into four main document divisions:\footnote{Here we try to avoid the
word {\em section}.}

\startitemize[n,packed]
\item front matter
\item body matter
\item appendices
\item back matter
\stopitemize

The document divisions are defined with:

\starttyping
\startfrontmatter ... \stopfrontmatter
\startbodymatter  ... \stopbodymatter
\startappendices  ... \stopappendices
\startbackmatter  ... \stopbackmatter
\stoptyping

The chapters in your book can be divided over these divisions.

\startbuffer
\starttext

\startstandardmakeup
  \midaligned{From Hasselt to America}
  \midaligned{by}
  \midaligned{J. Jonker and C. van Marle}
\stopstandardmakeup

\startfrontmatter

    \title{Preface}

    \chapter{Introduction}

\stopfrontmatter

\startbodymatter

    \chapter{The Rensselaer family}

    \chapter{The Lansing family}

    \chapter{The Cuyler family}

\stopbodymatter

\startappendices

    \chapter{Photos}

\stopappendices

\stoptext
\stopbuffer

\typebuffer

In the front matter as well as back matter the command \type{\chapter}
produces an un-numbered header in the table of contents. The front matter is mostly
used for the table of contents, the list of figures and tables, the preface, the
acknowledgements etc. It often comes with a roman page numbering.

The appendices division is used for (indeed) appendices. Headers may be typeset in
a different way; for example, \type{\chapter} may be numbered alphabetically.

The style of each document division can be set up with:

\shortsetup{setupsectionblock}

\stopchapter

\stopcomponent

