\environment texit-style

\startcomponent texit-conditions

\startchapter[title={Conditions}]

In case you wonder why we have modes in \CONTEXT, here is an example that might
convince you. The \TEX\ language has conditionals and they are in fact quite
efficient, take for instance:

\startTEX
\ifnum\scratchcounter>10
    \ifdim\scratchdimen>10pt
        one
    \else
        two
    \fi
\else
    three
\fi
\stopTEX

When the first test fails, \TEX\ will do a fast scan over the following tokens
and expand the \type {three} branch. In order to do such a fast scan, the nested
condition needs to be properly balanced: the \type {\else} is optional but the
nested \type {\fi} definitely isn't. Now imagine that you use a few pseudo
booleans, like:

\startTEX
\newif\ifalpha \alphatrue
\newif\ifbeta  \betatrue
\stopTEX

And you need it in:

\startTEX
\ifalpha
    \ifbeta
        YES
    \else
        NOP
    \fi
\else
    NOP
\fi
\stopTEX

This happens occasionally in real applications and one can either repeat the
\type {NOP} or wrap it in a macro in order to save tokens. However, way more
natural would be something like this:

\startTEX
\ifalphaorbeta
    YES
\else
    NOP
\fi
\stopTEX

This basically would introduce a new kind concept: an expandable macro flagged as
\type {\if} kind of token. I actually experimented with that in \LUATEX\ but
rejected it eventually. Instead \type {\ifcondition} was introduced. This is
basically equivalent to \type {\iffalse} when \TEX\ is in fast \type {\if*}
skipping mode, but when a real test happens the next argument is expanded. That
macro is expected to end up as something equivalent to \type {\iftrue} or \type
{\iffalse} so that other the nexct branch or the \type {\else} is entered. Here
is an example:

\startTEX
\ifcondition\alphaorbeta
    YES
\else
    NOP
\fi
\stopTEX

There are several ways to define \type {\alphaorbeta} now and we show a few here.
It's up to you to figure out which ons is the most efficient.

\startTEX
\def\alphaorbeta{\ifcase0\ifalpha \else\ifbeta \else1\fi\fi\relax}
\def\alphaorbeta{\ifcase \ifalpha0\else\ifbeta0\else1\fi\fi\relax}
\def\alphaorbeta{\ifnum1=\ifalpha1\else\ifbeta1\else0\fi\fi\relax}
\def\alphaorbeta{\ifnum 0\ifalpha1\fi  \ifbeta1\fi       >1\relax}
\stopTEX

Now, do we expect users to come up with such constructs? Of course not. Even in
\CONTEXT\ we don't really need them, although there are a few places where they
can be used. In \CONTEXT\ you would just do this:

\startTEX
\enablemode[alpha]
\enablemode[beta]

\doifelsemode {alpha,beta} {
    YES
} {
    NOP
}
\stopTEX

Of course such a verbose macro is less efficient but even if you use this test
10.000 times in a run it will not take more than 0.06 seconds on a decent 2013
laptop.

\stopchapter

\stopcomponent
