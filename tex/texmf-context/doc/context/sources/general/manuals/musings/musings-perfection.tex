% language=us runpath=texruns:manuals/musings

\startcomponent musings-perfection

\environment musings-style

\startchapter[title={Do we need perfection ?}]

\startsection[title={Introduction}]

It doesn't take much imagination to think of a future were jobs become less
interesting. Automation and especially machine learning will dominate more and
more. At least in the Netherlands there is a tendency to talk in terms of jobs
that assume \quotation {highly educated} folk, but often that is just a way to
boost one's self esteem, of worse, to suggest that a job is very interesting. In
this perspective the term {\em bs job} comes up. I like working with computers
but I'm not sure if I'd choose to work with them when I were young: the
perspective often for instance {\em scrumming through the workload} doesn't
really attract me. Add to that the fact that todays employer tomorrow has a
different owner and changed objectives and it becomes even less attractive.

I often get the impression that the more constructive jobs are also the more
interesting ones but alas they are not that advocated (or popular). On the
Internet you can find plenty videos of the amazing things that humans can do.
Incredible mechanical solutions: buildings, ships, planes, gears, waterworks,
chips, and when you go a bit back in time you'll notice that often what we
consider advanced today was inspired by the past (hint: search for lunar module
computer and colossus if you're interested in the origins of computing as we know
it today).

When you read about old school typography it is clear that we are dealing with a
craft. At some point extensive manual labor got assistance from tools: chisels,
lead letter forms, the printing press, semi|-|automated mechanical devices like
Monotype and Linotype but it was still craftmanship that was in demand. These
tools made it possible to scale up.

When computers came around the landscape changed. The things that could not be
automated still demanded manual intervention but soon (at least that is what we
noticed) the demands simply changed. When the computer cannot hyphenate well,
just forget about it and there's always an argument to come up with. And yes,
much of today's typesetting is not really new but comes from the (near) past. The
Monotype 4|-|line system for math looks quite interesting and advanced. If you
think that our emoji are hip and modern, just look at what the Aztec did.

The \TEX\ system has always been a bit different because it demands some manual
work to get things right. Of course one can use some precooked style but then
we're not talking craftmanship. The focus is then on the content and if it looks
kind of right all is good. It either gets retypeset \quotation {far far away} or
when it is just pushed on the web no one bothers (it has best fit on a phone).

It is really puzzling to see how little attention is paid to digitizing documents
It is not too hard to find 150 dpi scans where the pages were scanned in an angle
of relatively recent documents \typ
{https://www.tug.org/docs/liang/liang-thesis.pdf}. When I finally decided to buy
the original \CCODE\ book, a wondered why I had to pay some 60 dollar for what
looks like either a bad scan or some low quality digital print and after a first
glance I decided that I'll probably throw it in the paper bin some day soon
because I can as well use some bad scan from Internet. One of my first buys with
respect to typesetting was \quotation {Digital Typography: An Introduction to
Type and Composition for Computer System Design} by Richard Rubinstein. That one
was done on a Mac with MSWord in 1988 and looks better than the average document
done by \TEX\ that students get from their teachers who were (probably) educated
around that time. It's more about paying attention than about the tools.

The idea behind \CONTEXT\ macro package is (and will be) that users themselves
have some influence on what it does. There is no way that it can compete in the
\TEX\ domain with precooked styles simply because the standard has been set very
early to \LATEX\ (and \AMS\ math) and most marketing in the community targets to
such usage. One|-|time and one|-|shot users, of whom I bet some {\em have to} use
\TEX, but would gladly use something else, are not the audience for \CONTEXT. So
that brings us to questions like \quotation {What are the objectives of a macro
package like \CONTEXT ?}, \quotation {Do we need to provide perfect solutions.},
\quotation {What do users want?}. Being forced to use (a) \TEX\ (macro package)
is not much different from being forced to use an operating system, editor or
programming language. In the end there has to be a \quote {click} and when it's
there long term happy usage will often also lead to mastery and satisfaction.

Another set of questions relate to how eager we should be to keep up with all
demands. Do we really need to listen to publishers, especially when there is no
real indication of interest in typesetting but more in growth, profit, periodical
selling oneself. Demands change and standards come and go. A good example of a
currently popular demand is tagging documents, but when you look at that from
that with \TEX\ glasses on it is kind of weird. A \TEX\ based system can pretty
well render any variant of a document and target specific needs: paper, screen,
dedicated devices, different fonts, colors etc.\ is all pretty trivial. If a
house is not well accessible you adapt it. If clothing doesn't fit, you shop for
a different make. If you have an allergy you get different food. When it comes to
documents we can distribute variants and even the source. Not doing that when
there is demand is just laziness, unwillingness or going cheap. So, should we
adapt, or should we be more creative with designs and target alternative (and
multiple) media? If we don't, at least we should come up with good reasons.

To go back to what I started with: how do we keep working with documents, coming
up with solutions, playing with layout, interesting? It might as well be that in
the future when the ratio employment versus free time also gives more room for
figuring things out. The interaction between coming with the content and somehow
present it using tools like \CONTEXT\ can have a positive result on the final
product. Isn't is a nice challenge to serve a broad audience with well tuned
documents?

My impression is that the more extensive \CONTEXT\ users have enough freedom to
use a system like that. They are probably not working at large companies or in
large organizations that impress tools and methods. So a valid question is: how
does a system have to look like in order to draw those users into the game and
how does it keep them in a position that they can keep using it. We don't need
perfect, all automated, human replacement tools, do we? We're more talking
\quote {toolkit}, aren't we?

I'd like users to come up with additional sections here. How do they use
\CONTEXT ? What do they expect from a system like that? How should it evolve?
Where should it stop? What challenges should it leave to the user? What can go
and what (kind of control) should be added?

\stopsection

\startsection[title={Your turn}]

{\em user contributions}

\stopsection

\stopchapter

\stopcomponent
