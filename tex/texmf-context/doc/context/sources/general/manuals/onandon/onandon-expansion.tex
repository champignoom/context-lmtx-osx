% language=us

\startcomponent onandon-expansion

\environment onandon-environment

\startchapter[title={More (new) expansion trickery}]

Contrary to what one might expect when looking at macro definitions, \TEX\ is
pretty efficient. Occasionally I wonder if some extra built in functionality
could help me write better code but when you program with a bit care there is
often not much to gain in terms of tokens and performance. \footnote {The long
trip to the yearly Bacho\TeX\ meeting is always a good opportunity to ponder
\TEX\ and its features. The new functionality discussed here is a side effect of
the most recent trip.} Also, some possible extensions probably only would be
applied a few times which makes them low priority. When you look at the
extensions brought by \ETEX\ the number is not that large, and \LUATEX\ only
added a few that deal with the language, for instance \tex {expanded} which is
like an \tex {edef} without the defining a macro and acts on a token list wrapped
in (normally) curly braces. Just as reference we mention some of the expansion
related helpers.

\starttabulate[|l|l|p|]
\BC command \BC argument \BC
    comment
\NC \NR
\HL
\NC \tex {expandafter} \NC \type {token} \NC
    The token after the next token gets expanded (one level only). In tricky
    \TEX\ code you can often see multiple such commands in sequence which makes a
    nice puzzle.
\NC \NR
\NC \tex {noexpand} \NC \type {token} \NC
    The token after this command is not expanded in the context of expansion.
\NC \NR
\NC \tex {expanded} \NC \type {{tokens}} \NC
    The given token list is expanded. This command showed up early in \LUATEX\
    development and was taken from \ETEX\ follow|-|ups. I have mails from 2011
    mentioning its presence in \PDFTEX\ 1.50 (which was targeted in 2008) but
    somehow it never ended up in a production version at that time (and we're
    still not at that version). In \CONTEXT\ we already had a command with that
    name so there we use \tex {normalexpanded}. Users normally can just use the
    \CONTEXT\ variant of \type {\expanded}.
\NC \NR
\NC \tex {unexpanded} \NC \type {{tokens}} \NC
    The given token list is hidden from expansion. Again, in \CONTEXT\ we already
    had a command serving as prefix for definitions so instead we use \tex
    {normalunexpanded}. In the core of \CONTEXT\ this new \ETEX\ command is hardly
    used.
\NC \NR
\NC \tex {detokenize} \NC \type {{tokens}} \NC
    The given tokenlist becomes (basically) verbatim \TEX\ code. We had something
    like that in \CONTEXT\ but have no nameclash. It is used in a few places. It's
    also an \ETEX\ command.
\NC \NR
\NC \tex {scantokens} \NC \type {{tokens}} \NC
    This primitive interprets its argument as a pseudo file. We don't really use it.
\NC \NR %
\NC \tex {scantextokens} \NC \type {{tokens}} \NC
    This \LUATEX\ primitive does the same but has no end|-|of|-|file side
    effects. This one is also not really used in \CONTEXT.
\NC \NR
\NC \tex {protected} \NC \type {\.def} \NC
    The definition following this prefix, introduced in \ETEX, is unexpandable in
    the context of expansion. We already used such a command in \CONTEXT\ but
    with a completely different meaning so use \tex {normalprotected} as prefix
    or \tex {unexpanded} which is an alias.
\NC \NR
\stoptabulate

Here I will present two other extensions in \LUATEX\ that can come in handy, and
they are there simply because their effect can hardly be realized otherwise
(never say never in \TEX). One has to do with immediately applying a definition,
the other with user defined conditions. The first one relates directly to
expansion, the second one concerns conditions and relates more to parsing
branches which on purpose avoids expansion.

{\em In the meantime \LUAMETATEX\ has a slightly different implementation which
goes under the umbrella \quote {local control}. We show both ways here. The
example where two token lists are compared can be done easier with \type
{\iftok}.}

For the first one I use some silly examples. I must admit that although I can
envision useful application, I really need to go over the large amount of
\CONTEXT\ source code to really find a place where it is making things better.
Take the following definitions:

\startbuffer
\newcount\NumberOfCalls

\def\TestMe{\advance\NumberOfCalls1 }

\edef\Tested{\TestMe foo:\the\NumberOfCalls}
\edef\Tested{\TestMe foo:\the\NumberOfCalls}
\edef\Tested{\TestMe foo:\the\NumberOfCalls}

\meaning\Tested
\stopbuffer

\typebuffer

The result is a macro \tex {Tested} that not only has the unexpanded incrementing
code in its body but also hasn't done any advancing:

\getbuffer

Of course when you're typesetting something, this kind of expansion normally is
not needed. Instead of the above definition we can define \tex {TestMe} in a way
that expands the assignment immediately. You need of course to be aware of
preventing look ahead interference by using a space or \tex {relax} (often an
expression works better as it doesn't leave an \tex {relax}).

\startbuffer
% luatex

\def\TestMe{\immediateassignment\advance\NumberOfCalls1 }

% luametatex

\def\TestMe{\localcontrolled{\advance\NumberOfCalls1 }}

\edef\Tested{\TestMe bar:\the\NumberOfCalls}
\edef\Tested{\TestMe bar:\the\NumberOfCalls}
\edef\Tested{\TestMe bar:\the\NumberOfCalls}

\meaning\Tested
\stopbuffer

\typebuffer

This time the counter gets updated and we don't see interference in the resulting
\tex {Tested} macro:

\getbuffer

Here is a somewhat silly example of an expanded comparison of two \quote
{strings}:

\startbuffer
% luatex

\def\ExpandedDoifElse#1#2#3#4%
  {\immediateassignment\edef\tempa{#1}%
   \immediateassignment\edef\tempb{#2}%
   \ifx\tempa\tempb
     \immediateassignment\def\next{#3}%
   \else
     \immediateassignment\def\next{#4}%
   \fi
   \next}

% luametatex

\def\ExpandedDoifElse#1#2#3#4%
  {\localcontrolled{\edef\tempa{#1}}%
   \localcontrolled{\edef\tempb{#2}}%
   \ifx\tempa\tempb
     \localcontrolled{\def\next{#3}}%
   \else
     \localcontrolled{\def\next{#4}}%
   \fi
   \next}

\edef\Tested
  {(\ExpandedDoifElse{abc}{def}{yes}{nop}/%
    \ExpandedDoifElse{abc}{abc}{yes}{nop})}

\meaning\Tested
\stopbuffer

\typebuffer

I don't remember many cases where I needed such an expanded comparison. We have a
variant in \CONTEXT\ that uses \LUA\ but that one is not really used in the core.
Anyway, the above code gives:

\getbuffer

You can do the same assignments as in preambles of \tex {halign} and after \tex
{accent} which means that assignments to box registers are blocked (boxing
involves grouping and delayed assignments and so). The error you will get when
you use a non||assignment command refers to a prefix, because internally such
commands are called prefixed commands. Leading spaces and \tex {relax} are
ignored.

In addition to this one|-|time immediate assignment a pseudo token list variant
is provided, so the above could be rewritten to:

\starttyping
% luatex

\def\ExpandedDoifElse#1#2#3#4%
  {\immediateassigned {
     \edef\tempa{#1}
     \edef\tempb{#2}
   }%
   \ifx\tempa\tempb
     \immediateassignment\def\next{#3}%
   \else
     \immediateassignment\def\next{#4}%
   \fi
   \next}

% luametatex

\def\ExpandedDoifElse#1#2#3#4%
  {\beginlocalcontrol
     \edef\tempa{#1}
     \edef\tempb{#2}
   \endlocalcontrol
   \ifx\tempa\tempb
     \localcontrolled{\def\next{#3}}%
   \else
     \localcontrolled{\def\next{#4}}%
   \fi
   \next}
\stoptyping

While \tex {expanded} first builds a token lists that then gets used, the \tex
{immediateassigned} primitive just walls over the list delimited by curly braces.

A next extension concerns conditions. If you have done a bit of extensive \TEX\
programming you know that nested conditions need to be properly constructed in
for instance macro bodies. This is because (for good reason) \TEX\ goes into a
fast scanning mode when there is a match and it has to skip the \tex {else} upto
\tex {fi} branch. In order to do that properly a nested \tex {if} in there needs
to have a matching \tex {fi}.

In practice this is no real problem and careful coding will never give a problem
here: you can either hide nested code in a macro or somehow jump over nested
conditions if really needed. Actually you only need to care when you pickup a
token inside the branch because likely you don't want to pick up for instance a
\tex {fi} but something that comes after it. Say that we have a sane conditional
setup like this:

\starttyping
\newif\iffoo \foofalse
\newif\ifbar \bartrue

\ifoo
  \ifbar \else \fi
\else
  \ifbar \else \fi
\fi
\stoptyping

Here the \tex {iffoo} and \tex {ifbar} need to be equivalent to \tex {iftrue} or
\tex {iffalse} in order to succeed well and that is what for instance \tex
{footrue} and \tex {foofalse} will do: change the meaning of \tex {iffoo}.

But imagine that you want something more complex. You want for instance to let
\tex {ifbar} do some calculations. In that case you want it to behave a bit like
what a so called \type {vardef} in \METAPOST\ does: the end result is what
matters. Now, because \TEX\ macros often are a complex mix of expandable and
non|-|expandable this is not that trivial. One solution is a dedicated definer,
say \tex {cdef} for defining a macro with conditional properties. I actually
implemented such a definer a few years ago but left it so long in a folder with
ideas that I only found it back after I had come up with another solution. It was
probably proof that it was not that good an idea.

The solution implemented in \LUATEX\ is just a special case of a test: \tex
{ifcondition}. When looking at the next example, keep in mind that from the
perspective of \TEX's scanner it only needs to know if something is a token that
does some test and has a matching \tex {fi}. For that purpose you can consider
\tex {ifcondition} to be \tex {iftrue}. When \TEX\ actually wants to do a test,
which is the case in the true branch, then it will simply ignore this \tex
{ifcondition} primitive and expands what comes after it (which is \TEX's natural
behaviour). Effectively \tex {ifcondition} has no meaning except from when it has
to be skipped, in which case it's a token flagged as \tex {if} kind of command.

\starttyping
\unexpanded\def\something#1#2%
  {\edef\tempa{#1}%
   \edef\tempb{#2}
   \ifx\tempa\tempb}

\ifcondition\something{a}{b}%
    \ifcondition\something{a}{a}%
        true 1
    \else
        false 1
    \fi
\else
    \ifcondition\something{a}{a}%
        true 2
    \else
        false 2
    \fi
\fi
\stoptyping

Wrapped in a macro you can actually make this fully expandable when you use the
previously mentioned immediate assignment. Here is another example:

\starttyping
\unexpanded\def\onoddpage
  {\ifodd\count0 }

\ifcondition\onoddpage odd \else even \fi page
\stoptyping

The previously defined comparison macro can now be rewritten as:

\starttyping
% luatex

\def\EqualTokens#1#2%
  {\immediateassignment\edef\tempa{#1}%
   \immediateassignment\edef\tempb{#2}%
   \ifx\tempa\tempb}

\def\ExpandedDoifElse#1#2#3#4%
  {\ifcondition\EqualTokens{#1}{#2}%
     \immediateassignment\def\next{#3}%
   \else
     \immediateassignment\def\next{#4}%
   \fi
   \next}

% luametatex

\def\EqualTokens#1#2%
  {\localcontrolled{\edef\tempa{#1}}%
   \localcontrolled{\edef\tempb{#2}}%
   \ifx\tempa\tempb}

\def\ExpandedDoifElse#1#2#3#4%
  {\ifcondition\EqualTokens{#1}{#2}%
     \localcontrolled{\def\next{#3}}%
   \else
     \localcontrolled{\def\next{#4}}%
   \fi
   \next}
\stoptyping

When used this way it will of course also work without the \tex {ifcondition} but
when used nested it can be like this. This last example also demonstrates that
this feature probably only makes sense in more complicated cases where more work
is done in the \tex {onoddpage} or \tex {equaltokens} macro. And again, I am not
sure if for instance in \CONTEXT\ I have a real use for it because there are only
a few cases where nesting like this could benefit. I did some tests with a low
level macro where it made the code look nicer. It was actually a bit faster but
most core macros are not called that often. Although the overhead of this feature
can be neglected, performance should not be the reason for using it: in \CONTEXT\
for instance one can often only measure such possible speed|-|ups on macros that
are called tens or hundreds of thousands of times and that seldom happens in a
real run end even then a change from say 0.827 seconds to 0.815 seconds for 10K
calls of a complex case is just noise as the opposite can also happen.

Although not strictly necessary these extensions might make some code look better
so that is why they officially will be available in the 1.09 release of \LUATEX\
in fall 2018. It might eventually inspire me to go over some code and see where I
can improve the look and feel.

The last few years I have implemented some more ideas as local experiments, for
instance \tex {futurelet} variant or a simple (one level) \tex {expand}, but in
the end rejected them because there is no real benefit in them (no better looking
code, no gain in performance, hard to document, possible side effects, etc.), so
it is very unlikely that we will have more extensions like this. After all, we
could do more than 40 years without them. Although \unknown\ who knows what we
will provide in \LUATEX\ version~2.

\stopchapter

\stopcomponent
