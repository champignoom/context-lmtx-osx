% macros=mkvi

% \unprotect \pushoverloadmode
% \protect

\setupbackend[format=pdf/ua-2]
% \setuptagging[state=start]
% \nopdfcompression
% \enabletrackers[structures.tags]
% \enabletrackers[structures.tags.showtree]

\environment examples-style
\environment examples-style-math

% todo-group.txt

% \setupnote[mathnote][location=page]
% \enabletrackers[math.textblobs]

% \disabledirectives[structures.tags.shipout]
% \enabledirectives [structures.tags.math.standalone]
% \disabledirectives[structures.tags.math.strip]

% \setuptagging
%   [state=start]

\definemathgroupset
%   [demob]
  [every]

% \setmathgroupset
%   [demob]

\registermathfunction[𝑓]
\registermathfunction[𝑔]

% \registermathsymbol[default][en][𝐮][the vector]
% \registermathsymbol[default][en][𝐯][the vector]
% \registermathsymbol[default][en][𝖠][the matrix]

\registermathsymbol[default][en][lowercasebold]           [the vector] % [of]
\registermathsymbol[default][en][uppercasesansserifnormal][the matrix]

% \registermathsymbol[default][sv][𝐮][vektorn]
% \registermathsymbol[default][sv][𝐯][vektorn]

\registermathsymbol[default][sv][lowercasebold]           [vektorn]
\registermathsymbol[default][sv][uppercasesansserifnormal][matrisen]

% \registermathsymbol[default][en][𝒞][the continuous functions]
% \registermathsymbol[default][sv][𝒞][de kontinuerliga funktionerna]

        \registermathsymbol[default][ua][lowercasebold]           [vektorn]
        \registermathsymbol[default][ua][uppercasesansserifnormal][matrisen]

%         \registermathsymbol[default][de][lowercasebold]           [vektorn]
%         \registermathsymbol[default][de][uppercasesansserifnormal][matrisen]

\def\ExampleLanguages{en,sv}
\def\ExampleLanguages{debug,en,sv}
% \def\ExampleLanguages{debug,en,sv,nl}
% \def\ExampleLanguages{debug,en,sv,ua}
% \def\ExampleLanguages{en,sv,ua}
% \def\ExampleLanguages{en,sv,ua,de}

% \enablemode[issues]

% \enabletrackers[structures.tags.math.times]
% \enabletrackers[structures.tags.math.keeplast=mmldata]
% \enabletrackers[structures.tags.math.save]

% The document itself:

\startbuffer [colophon]

\startsubject[title=About this document]

This document is used by Mikael Sundqvist and Hans Hagen to check out how well a
formula translates to a verbose meaning. It's an experiment with accessibility on
the one hand but also a way to get documents validated and even annotated.
Eventually there will be support for many languages but we started with English,
Swedish and Dutch.

\blank

This feature is only available in \CONTEXT\ \MKXL, aka \LMTX. You can enable
tracking in your document by for instance:

\starttyping[option=TEX]
\setuptagging
  [state=start]

\definemathgroupset
  [mydemogroup]
  [every]

\setmathgroupset
  [mydemogroup]

\setupnote
  [mathnote]
  [location=page]

\enabletrackers
  [math.textblobs]
\stoptyping

By default a \type {mathnote} is set up to be an endnote in which case you need
to place them with:

\starttyping[option=TEX]
\placenote[mathnode]
\stoptyping

\stopsubject
\stopbuffer

\startdocument
  [title={Meaningfull Math},
   author={Mikael Sundqvist & Hans Hagen}]


\StartExample
    % Addition and equals
    \im {1 + 2 = 3}
\StopExample

\StartExample
    % Subtraction and negative number
    \im {1 - 2 = -1}
\StopExample

\StartExample
    % Multiplication
    \im {2 \times 3 = 2 \cdot 3 = 6}
\StopExample

\StartExample
    % Decimal numbers
    \im {1.1 + 2.22 = 3.33 = 3 + (0.1 + .22) \neq - \digits{1.23^4} \neq 10^5}
\StopExample

\StartExample
    % Hexadecimal with \mn
    \im {\mn{0x34BE} = 13502 = \digits{13502}}
\StopExample

\StartExample
    % Squared
    \im {3^2 + 4^2 = 5^2}
\StopExample

\StartExample
    % Higher power
    \im {3^4 + 4^4 \neq 5^4}
\StopExample

\StartExample
    % Simple fraction
    \dm {\frac{1}{2} = \frac{1}{3} + \frac{1}{6}}
\StopExample

\StartExample
    % Fraction with symbols
    \dm {\frac{1}{x} + \frac{1}{y} = \frac{x + y}{xy}}
\StopExample

\StartExample
    % Fraction multiplied by number
    \m{\frac{1}{2}2 = \frac{1}{2} \cdot 2 = \frac{1}{2} \times 2 = 2 \frac{1}{2} = 2 \cdot \frac{1}{2} = 2 \times \frac{1}{2}}
\StopExample

\StartExample
    % Fraction multiplied by symbol
    \m{\frac{1}{2}a = \frac{1}{2}\cdot a = \frac{1}{2} \times a = a \frac{1}{2} = a \cdot \frac{1}{2} = a \times \frac{1}{2}}
\StopExample

\StartExample
    % With fraction times fraction
    \dm {a\frac{1 + x}{x - 1} + \frac{1 - x}{1 + x}\frac{1 - y}{1 + y} + \frac{1 - x}{1 + x}y}
\StopExample

\StartExample
    % Group and number/variable
    \im {2(1 + x) + (1 + y)3 - a(1 + z) - (1 + u)b}
\StopExample

\StartExample
    % Group and number/variable with explicit multiplication
    \im {2 \cdot (1 + x) + (1 + y) \cdot 3 - a \cdot (1 + z) - (1 + u) \cdot b}
\StopExample

\StartExample
    % Multiplication of indexed/sub (use \notimes if times should be surpressed)
    \dm {a__2b__1 - a_1b_2 = a__2\notimes b__1 - a_1\notimes b_2}
\StopExample

\StartExample
    % A few indices, both as one and as multi
    % Do we want to use invisible comma anywhere? Probably not.
    \dm {A_{1,20} + A_1_{20} + A_{1,20} + A_1_{20}}
\StopExample

\StartExample
    % Group and element with sub(index)
    \im {a__1(1 + x) + (1 + y)b__1 - a_2(1 + z) - (1 + u)b_2}
\StopExample

\StartExample
    % Group and element with sub(index) and with explicit multiplication
    \im {a__1 \cdot (1 + x) + (1 + y) \cdot b__1 - a_2 \cdot (1 + z) - (1 + u) \cdot b_2}
\StopExample

\StartExample
    % Groups and times
    \im {(n+k)n + (n+k)(n+1) + n(n+k)}
\StopExample

\StartExample
    % Left right groups times
    \im {\left(n+k\right)n + \left(n+k\right)\left(n+1\right) + n\left(n+k\right)}
\StopExample

\StartExample
    % Fenced and times
    \im {\fenced[parenthesis]{n+k}n + \fenced[parenthesis]{n+k}\fenced[parenthesis]{n+1} + n\fenced[parenthesis]{n+k}}
\StopExample

\StartExample
    % Groups with powers and times
    \dm {(1 + x)^n a  = a (1 + x)^n \neq (1 + x)^n (1 + y) - (1 + x)(1 + y)^n}
\StopExample

\StartExample
    % Simple parenthesis usage
    % Better use structured input (see next two examples)
    \im {(1 + 2 + 3 + 4)^2 = 1^3 + 2^3 + 3^3 + 4^3}
\StopExample

\StartExample
    % Better, but next one might be even more clear
    \im {\left(1 + 2 + 3 + 4 \right)^2 = 1^3 + 2^3 + 3^3 + 4^3}
\StopExample

\StartExample
    % Structured parenthesis usage
    \im{\fenced[parenthesis]{1 + 2 + 3 + 4 }^2 = 1^3 + 2^3 + 3^3 + 4^3}
\StopExample

\StartExample
    % Plus minus
    \im {x \neq x + -1}
\StopExample

\StartExample
    % Decimal period last in math goes
    \im {x = 1.}
\StopExample


\StartExample
    % Also goes for mathtextpunctuation
    % (Mostly for displayed formulas, otherwise, keep the punctuation outside math)
    \im {x = 1\mtp{.}}
\StopExample

\StartExample
    % Period at end -> period goes
    \im {x = y.}
\StopExample

\StartExample
    % Also goes for mathtextpunctuation
    \im {x=z\mtp{.}}
\StopExample

\StartExample
    % Shoulf comma at the end also go? (bad input)
    \im {x = y,}
\StopExample

\StartExample
    % Variables can be used as placeholders for numbers (explaining decimals)
    % We use \notimes to get rid of the explicit multiplication
    \im {a__{0}.a__{1}\notimes a__{2} \ldots a__{n} \ldots}
\StopExample

\StartExample
    % Different ways to access the multiplication dot
    \im {y·z = y \cdot z = y \scalarproduct z}
\StopExample

\StartExample
    % The f and g are in this document registered as functions
    % There should be TIMES between g and h
    \im {abcdefghikl}
\StopExample

\StartExample
    % Lots of times
    \im {xx \sin(x) x \frac{x}{x} x \sqrt{x} x \int x \sin \cos x \sin(x) \cos}
\StopExample

\StartExample
    % f is registered as a function, h is not
    \im {af(x) + bh(x) + f(x + b)}
\StopExample

\StartExample
    % Apply function or whatever
    \dm { A(X) \neq A\notimes(X) \neq A\applyfunction(X) \neq A\of(X)}
\StopExample

\StartExample
    % Just an example where \of makes sense
    \dm {\Sigma \of (X \vee Y) = \Sigma \of X \vee \Sigma \of Y}
\StopExample

\StartExample
    % An example with something of two variables
    \dm { F\of(x,t) = f__t(x) = \mathrm{f}__t\of(x)}
\StopExample

\StartExample
    % Prime with and without \of
    \dm { h'\of(x) \neq h'(x)}
\StopExample

\StartExample
    % C \of examples
    % We shall not get rid of the grouping since it gives structure
    % One could think of a \nogroup (just as \notimes)
    \im { C \of (\openinterval{a,b}) \neq C^^2 \of (\interval{a,b}) \neq C^^2 \of \interval{0,1} \neq C\of(\Omega) \neq 𝒞 \of (\Omega)}
\StopExample

\StartExample
    % Nesting groups. Could it have meaning? Or should we only get one group.
    \im { (((x))) \neq ((x)) \neq (\parenthesis{x})}
\StopExample

\StartExample
    % Nesting groups/parentheses need to be there
    \im { s\of(1) = s\of(\set{0}) = \set{0} \cup \set{\set{0}}}
\StopExample

\StartExample
    % Algebra (ring) examples
    \dm {\reals \of \bracket{x + 1} = \reals\fenced[bracket]{x} \neq \reals\of[x]}
\StopExample

\StartExample
    % This is a result of
    % \registermathsymbol[default][en][lowercasebold][the vector]
    % \registermathsymbol[default][en][uppercasesansserifnormal][the matrix]
    \im {\mathss{A}__{\mathbf{u}}\mathbf{v} \colonequals \mathbf{u} \crossproduct \mathbf{v}}
\StopExample

\StartExample
    % Binomials are fractions
    \im {\binom{3}{2} = \frac{3!}{(3-2)!2!}}
\StopExample

\StartExample
    % With symbols it gets a bit long
    \im {\binom{2n}{n + 1} = \frac{(2n)!}{(n - 1)!(n + 1)!}}
\StopExample

\StartExample
    % Binomials, multiplied
    \im {a\binom{n}{k} + \binom{n}{k}\binom{n}{a} + \binom{n}{k}x^k + \binom{n}{k}x}
\StopExample

\StartExample
    % Binomial theorem
    \dm {\parenthesis{1 + x}^n = \sum_{k = 0}^{n} \binom{n}{k}x^k}
\StopExample

\StartExample
    % \ldots = , and so on
    \im {x + x^2 + x^3 + \ldots = x/(1 - x)}
\StopExample

\StartExample
    % Well-known complex formula
    \im {3i \neq 3\ii \neq 1 + i \neq 2 + \ii \neq 3 + a i \neq 3 + a \ii }
\StopExample

\StartExample
    % Well-known complex formula
    \im {\ee^{\pi \ii } = -1}
\StopExample

\StartExample
    % Do we need "times" before the \ee?
    \im {a + b \ii = \sqrt{a^2 + b^2}\ee^{\ii\arg(a + \ii b)}}
\StopExample

\StartExample
    % Simple conjugate
    \im {\conjugate{a + b \ii} = a - b \ii}
\StopExample

\StartExample
    % Implication
    \im {x^2 = -1 \implies x = \pm \ii}
\StopExample

\StartExample
    % Some radicals
    \im {\sqrt{x} = x^{1/2} \neq x^{1/3} = \root[3]{x}}
\StopExample

\StartExample
    % Some radicals with multiplication
    \im {2\sqrt{x} = 2x^{1/2} \neq 2x^{1/3} = 2\root[3]{x}}
\StopExample

\StartExample
    % Some radicals with multiplication
    \im {a\sqrt{x} = ax^{1/2} \neq ax^{1/3} = a\root[3]{x}}
\StopExample

\StartExample
    % Some radicals with multiplication
    % This is bad input!
    \im {\sqrt{x}2 = x^{1/2}2 \neq x^{1/3}2 = \root[3]{x}2}
\StopExample

\StartExample
    % Product of radicals
    \im {\sqrt{x} \sqrt{y} = \sqrt{xy}}
\StopExample

\StartExample
    % Some radicals with multiplication
    \im {\sqrt{x}a = x^{1/2}a \neq x^{1/3}a = \root[3]{x}a}
\StopExample

\StartExample
    % Just a few numbersets with subsets
    \im {\naturalnumbers \subset \integers \subset \rationals \subset \reals \subset \complexes}
\StopExample

\StartExample
    % Just a few numbersets with intersection
    \im {\naturalnumbers \cap \reals = \naturalnumbers}
\StopExample

\StartExample
    % A set with a \fence. Notice that no group should be started after the fence
    \im {\set{a \in \naturalnumbers \fence \mtext{\im{a} is even}}}
\StopExample

\StartExample
    % A set with a \fence. More conditions
    \dm {\rationals = \set{\frac{p}{q} \fence p,q \in \integers \land q \neq 0}}
\StopExample

\StartExample
    % Maps colon is given by \maps (defined function)
    \im {f \maps \reals \to \reals}
\StopExample

\StartExample
    % Maps colon is given by \maps (named function)
    \im {\sin \maps \reals \to \reals}
\StopExample

\StartExample
    % Maps as colon by \mapsas
    \im {f \mapsas x \mapsto x + \exp(x)}
\StopExample

\StartExample
    % Maps as colon by \mapsas
    \im {\sin \mapsas x \mapsto \sin(x)}
\StopExample

\StartExample
    % Logarithms, spelled out
    % Todo, add for other or remove for ln
    \im {x \mapsto \ln(x)}
\StopExample

\StartExample
    % The grouping is sometimes needed
    \im {\sin x = \sin(x) \neq \sin(x) + 1 \neq \sin(x + 1)}
\StopExample

\StartExample
    % Just a function
    \im {f = \sin}
\StopExample

\StartExample
    % Just a limit
    \im {\lim a_{k} = -\infty}
\StopExample

\StartExample
    % A limit with sub on lim
    \im {\lim_{k \tendsto +\infty} a_{k}}
\StopExample

\StartExample
    % Using index (__)
    \im {\lim__{k \tendsto +\infty} a__{k} = -\infty}
\StopExample

\StartExample
    % Limit and fractin (no times inbetween)
    \dm {\lim \frac{a__{k}}{b__{k}}}
\StopExample

\StartExample
    % Limit and fraction with sub on lim
    \dm {\lim_{k \tendsto +\infty} \frac{A__k}{B__k}}
\StopExample

\StartExample
    % Should be two formulas, but in this document we only show the last one
    \im {f(x) \tendsto A \mtext{ as } x \tendsto a}
\StopExample

\StartExample
    % Just a standard limit
    \dm {\lim_{x \tendsto 0} \frac{\sin (x)}{x} = 1}
\StopExample

\StartExample
    % More complicated in the sub.
    \dm {\lim_{f(x) \tendsto 0} g(x)}
\StopExample

\StartExample
    % Some derivatives
    % Do we want "The function" here? (That is a more general question)
    \im {f'(x) + f''(x) + f'''(x) + f''''(x)}
\StopExample

\StartExample
    % Variable primed
    \im {f' + h' + h'' + h''' + h''''}
\StopExample

\StartExample
    % More derivatives
    \im { \secondderivative{f} = f'' }
\StopExample

\StartExample
    % An example with derivative
    \im {\sin''(x) = -\sin(x) = \sin(x + \pi)}
\StopExample

\StartExample
    % Even more derivatives, also with indices
    \im {f__1'(x) + f__1^^2'(x) }
\StopExample

\StartExample
    % Without \notimes we get a times. See also next example
    \im {(f)'(x) + (f)'\notimes(x) + \derivative{(f)}(x) + \derivative{(f)}\notimes(x)}
\StopExample

\StartExample
    % Here we want times, so we cannot block it in previous example
    \im {(f+g)'(f+g) }
\StopExample

\StartExample
    % More indices
    \im {(f__1)^2 = (f_1)^2 \neq f_1^2 }
\StopExample

\StartExample
    % A few more
    \im {(x_1)^2 \neq x_1^2 }
\StopExample

\StartExample
    % More indiced, we probably can remove some
    \im {h_1 + h__1 + h^1 + h^^1}
\StopExample

\StartExample
    % Amazing multiscript example
    \im{
        h_{}      ^{\lambda} ___{s}
         _{\kappa}^{} % \noscript  %
         _{\mu}   ^{} % \noscript  %
         _{}      ^{\nu}     ___{t}
         _{\phi}  ^{} % \noscript
        }
\StopExample

\StartExample
    % Multiscripts
    \im {\Gamma_1^2_3^4 \neq \Gamma__1^^2__3^^4}
\StopExample

\StartExample
    % Even more multiscripts
    \im {\Gamma__1^^2__3^^4 \neq \Gamma__1^^2^^{}__3^^4}
\StopExample

\StartExample
    % One example with prescript
    The hypergeometric function \im {F____2__1}
\StopExample

\StartExample
    % A sum and a fraction
    \dm {\sum_{n = 1}^{+\infty} \frac{1}{n^2} = \frac{\pi^2}{6}}
\StopExample

\StartExample
    % A sum with only sub index, and a fraction
    \dm {\sum_{n \in \naturalnumbers} \frac{1}{n^2} = \frac{\pi^2}{6}}
\StopExample

\StartExample
    % A product followed by a delimitered parenthesis
    \dm {\sin x = \prod_{n = 1}^{+\infty} \left(1 - \frac{x^2}{\pi^2n^2}\right)}
\StopExample

\StartExample
    % A product followed by a fence
    \dm {\sin x = \prod_{n = 1}^{+\infty} \parenthesis{1 - \frac{x^2}{\pi^2n^2}}}
\StopExample

\StartExample
    % A simple integral with limits
    \dm {\int_{a}^{b} f'(x) \dd x = f(b) - f(a)}
\StopExample

\StartExample
    % A bit more complex lower limit
    \dm {\int_{x=a}^{b} f'(x) \dd x = f(b) - f(a)}
\StopExample

\StartExample
    % An integral over the domain
    \dm {\int_{\Omega} f \dd \mu = 0}
\StopExample

\StartExample
    % An integral followed by a fraction
    \dm {\int \frac{1}{1 + x^2} \dd x}
\StopExample

\StartExample
    % An integral with limits, followed by a fraction
    \dm {\int_0^1 \frac{1}{1 + x^2} \dd x}
\StopExample

\StartExample
    % Some tuples
    \im {\tuple{x^1, x^2, x^3} \neq \tuple{x^^1, x^^2, x^^3} = \tuple{x__1, x__2, x__3}}
\StopExample

\StartExample
    % Complement shall not give times
    \im {A \cup \complement A}
\StopExample

\StartExample
    % Quantifiers
    % We need to think about unary operators (class) in a broader sense
    \im {\forall x \in A \exists y \in B: \abs{x - y} > 1}
\StopExample

\StartExample
    % Right function adjoint
    \im {\adjoint{T}T = T\adjoint{T} \neq \adjoint{T}}
\StopExample

\StartExample
    % Same comment as for adjoint
    \im {A \adj(A) = \det(A) I}
\StopExample

\StartExample
    % Convolution with non-regiestered functions, note the \of
    \im {(u \convolve v) (x) \colonequals \int_{\reals} u\of(\xi) v\of(x - \xi) \dd \xi}
\StopExample

\StartExample
    % Convolution with registered functions
    \im {(f \convolve g) (x) \colonequals \int_{\reals} f(\xi) g(x - \xi) \dd \xi}
\StopExample

\StartExample
    % Right transpose function
    \im {\transpose{A} + \transpose{(A + B^2)} + \transpose{\left(A^2 + B\right)}}
\StopExample

\StartExample
    % Partial derivatives with lower indices. Beware of order.
    \im {\secondderivative{f__{xy}} = \secondderivative{f__{yx}} = f__{xy}'' \neq \secondderivative{f}__{yx}}
\StopExample

\StartExample
    % Inverse of function f
    % It is the preimage that is the issue
    \im {f(x) = y \iff x = \inverse{f}\of(y)}
\StopExample

\StartExample
    % Inverse of variable h
    \im {h\of(x) = y \iff x = \inverse{h}\of(y)}
\StopExample

\StartExample
    % Preimage of function f
    \im {\preimage{f}\of(Y) = \set{x \in X \fence f(x) = y}}
\StopExample

\StartExample
    % Preimage of variable h
    \im {\preimage{h}\of(Y) = \set{x \in X \fence h\of(x) = y}}
\StopExample

\StartExample
    % Leibniz derivatives
    \dm {\frac{\dd u}{\dd t} = u' = \dot{u}}
\StopExample

\StartExample
    % More derivatives, Laplace operator
    % Maybe we need operatorof here?
    \dm {\frac{\partial u}{\partial t} = c^2 \laplace u}
\StopExample

\StartExample
    \dm {\frac{\partial u}{\partial t} = c^2 \frac{\partial^2u}{\partial x^2}}
\StopExample

\StartExample
    % More derivatives
    % Here we see that "the function" from the registered
    % function is not always wanted
    \dm {\dd + \frac{\dd^3 u}{\dd x^3} + \frac{\dd f}{\dd x}}
\StopExample

\StartExample
    % Upright d
    \setupmathematics[differentiald=upright]
    \dm {\dd + \frac{\dd^3 u}{\dd x^3} + \frac{\dd f}{\dd x}}
\StopExample

\StartExample
    % A mixed partial derivative
    \dm {\frac{\partial^3 u}{\partial x^2 \partial y}}
\StopExample

\StartExample
    % A complex analysis way of writing it.
    % To be thought of
    \im {\conjugate{\partial} u = \bar{\partial} u = f}
\StopExample

\StartExample
    % Experimented with partial derivative d d x group u plus v end group...
    % but for accessibility reasons it is better to keep the partial
    % We need \of here because one can have products as well
    % Without \of one could consider \notimes, but if none is there we should get a TIMES
    \dm {\frac{\partial}{\partial x}\of(u + v) = \frac{\partial u}{\partial x} + \frac{\partial v}{\partial x}}
\StopExample

\StartExample
    % One example with Laplace followed by a close parenthesis
    \im {(1 - \laplace)u = f}
\StopExample

\StartExample
    % Just a few operators
    \im {\laplace = \gradient \scalarproduct \gradient = \gradient^2 = \nabla \scalarproduct \nabla}
\StopExample

\StartExample
    % Operator and crossproduct
    \im {\gradient \crossproduct \gradient}
\StopExample

\StartExample[issue]
    % Maybe shorter end
    % Made an issue to enable discussion
    \im {\floor{3.6} = \ceiling{2.7} = \integerpart{3.2}}
\StopExample

\StartExample
    % Just a set
    \im {A = \set[size=1]{1, 2, 3}}
\StopExample

\StartExample
    % Just a tuple
    \im {A = \tuple{1, 2, 3}}
\StopExample

\StartExample
    % Just an absolute value
    \im {|\abs[size=0]{a__{n__k} - A} < \epsilon}
\StopExample

\StartExample
    % Inner product
    \im {\innerproduct{u \fence v} = \conjugate{\innerproduct{v \fence u}}}
\StopExample

\StartExample
    % A set with a fence
    \im {\reals__{+} \colonequals \set{x \fence x \in \reals \land x > 0}}
\StopExample

\StartExample
    % Simple unstructured input works, but do not use!
    % \setupmathematics[autointervals=no]
    \im {[a,b[ \neq ]a,b] \neq ]a,b[ \neq [a,b]}
\StopExample

\StartExample
    % Warning: Nesting with weird parenthesis is not supported
    \im {X = \varleftopeninterval{a,(b + 1)} \neq ]a,(b + 1)]}
\StopExample

\StartExample
    % Closure of interval
    \im {\closure{\openinterval{a,b}} = \closedinterval{a,b}}
\StopExample

\StartExample
    % Closure of interval
    \im {\closure{\varopeninterval{0,1}} = \closedinterval{0,1}}
\StopExample

\StartExample
    % Random formula
    % Mikael: Think about the Delta
    \dm {u\of(b)-u\of(a)=\lim_{n\to+\infty} \parenthesis{f(x__1)\Delta x__1+f(x__2)\Delta x__2+\ldots+f(x__n)\Delta x__n}}
\StopExample

\StartExample
    % Absolute value, triangle inequality
    \im {\abs{x + y} \leq \abs{x} + \abs{y}}
\StopExample

\StartExample
    % Norm, triangle inequality
    \im {\norm{x + y} \leq \norm{x} + \norm{y}}
\StopExample

\StartExample
    % Both norm and absolute value
    \im {\norm{\alpha x} = \abs{\alpha} \norm{x}}
\StopExample

\StartExample
    % Example with \mtp
    \im {f(x) = x^2 \mtp{,} x \in \reals}
\StopExample

\StartExample
    % Logic example
    \im {\lnot(P \lor Q) = (\lnot P) \land (\lnot Q)}
\StopExample

\StartExample
    % \neg is now defined as a function. Maybe char-def
    % TODO Fix \neg?
    \im {\neg(P \land Q) \iff (\lnot P) \lor (\lnot Q)}
\StopExample

\StartExample
    % Yet another example with quantifier
    % Observe the usage of \notimes
    \im {(\forall x \in \reals)\notimes (x > 0 \lor x = 0 \lor x < 0)}
\StopExample

\StartExample
    % Cases example
    \dm {f(x) =
        \startcases
            \NC x  \NC x > 0 \NR
            \NC -x \NC x < 0 \NR
        \stopcases}
\StopExample


\StartExample
    % Cases with lefttext
    \dm{f(x) =
        \startcases[lefttext=\mtp{,}]
            \NC x  \NC x > 0 \NR
            \NC -x \NC x < 0 \NR
        \stopcases}
\StopExample

\StartExample
    % Cases with righttext
    \dm{f(x) =
        \startcases[righttext=\mtext{if }]
            \NC x  \NC x > 0 \NR
            \NC -x \NC x < 0 \NR
        \stopcases}
\StopExample

\StartExample
    % Chemistry example
    % Todo: maybe defaultstyle to \tf
    \setupmathematics[domain=chemistry]
    \dm{
        {\tf X}^^^{123}__{12}^{+4} \approx X^^^{123}__{12}^{+4}
    }
\StopExample

\StartExample
    \dm{
        \left(1 + x + x^2\right)^2
    }
\StopExample

\StartExample
    \setupmathematics[domain=simplified]
    \dm{
        \left(1 + x + x^2\right)^2
    }
\StopExample

\StartExample
    % User formula
    % Maybe add something inbetween nedsted sums (and integrals)?
    \startformula \chi^{2} = \sum_{i = 1}^{r}{\sum_{j = 1}^{c}\frac{\left( O_{ij} - E_{ij} \right)^{2}}{E_{ij}}} \stopformula
\StopExample

\stopdocument
