% language=us runpath=texruns:manuals/evenmore

\startcomponent evenmore-whattex

\environment evenmore-style

\startchapter[title={Is \LUAMETATEX\ still \TEX ?}]

\startsection[title={Introduction}]

Is \LUAMETATEX\ really a \TEX\ (compatible) engine? The answer to that depends on
how you define \TEX. If you think of the program with the same name, the answer
is definitely \quotation {no}, simply because a program that is not exactly
behaving like \quotation {\TEX\ The Program} cannot be called \TEX. This is why
derived programs have \type {tex} in their name but also some addition that
indicates that it isn't the original: \type {e}, \type {pdf}, \type {lua}. Don't
confuse that with macro package names that have \type {tex} in their name. If you
find such binaries that they are likely some stub to an engine (binary) that
preloads a format file (a memory dump) with the same name.

When you mean \quotation {\TEX\ The Macro Language} the answer is a bit more
nuanced especially when the results are pretty close to identical. In the next
sections I will discuss this in more detail from the perspective of how \CONTEXT\
evolved and what engines it has used.

\stopsection

\startsection[title={Multiple engines}]

When we started with \CONTEXT\ there was not that much choice in engines.
Basically one just used original \TEX, but although we used the version that came
with the book, pretty soon we switched to em\TEX, a version that gave more
memory; later a real huge version showed up. The fonts used were bitmaps and the
viewer was a \DVI\ bitmap viewer. However, when our new printer could not be set
up properly we decided to move on to \POSTSCRIPT\ fonts. That also meant using a
different backend driver (\DVIPSONE). And then of course we also started using a
previewer that could handle outline fonts. Once you start along that route
graphics come into play, color shows up and hyperlinks become an option. A couple
of years later the \PDF\ document rendering format was introduced. This paragraph
already mentions a lot of different programs and adaptations, but we're still
talking good old \TEX\ here and \CONTEXT\ was set up in such a way that it
adapted itself to whatever ecosystem made sense. When looking at \TEX\ one has to
consider the front as well as the backend, and both have related primitives and
features. Extensions to the frontend have been driven by the demands of macro
packages (beyond the original ideas) and those of the backend relate to what the
evolving rendering demands impose.

A couple of decades ago the \ETEX\ project started. It's objective was to extend
stable \TEX\ with a couple of more primitives and features: it is a superset and
therefore still \TEX, but as it really is an extension the name was extended too
(with the bit unusual character $\varepsilon$). At that point the main reason for
\CONTEXT\ was convenience because the new features were already kind of present
in the code base (think of emulated behavior). Again the macro package adapted
itself at runtime.

Then \PDFTEX\ came around which had some impact. It introduced the concept of a
built|-|in backend that avoided additional programs. The \ETEX\ extensions were
merged into this program so that basically meant that it replaced its
predecessors. For a user \PDFTEX\ was just \TEX. For some reason the narrative
became that \CONTEXT\ depended on \PDFTEX, probably because it was always quick
in using its features, a side effect of being close to the development.

The \CONTEXT\ package was an early adopter of \METAPOST\ and that graphic
subsystem, although still external, was integrated in such a way that users could
think of it being embedded. This was made possible by the fact that right from
the start \CONTEXT\ came with an infrastructure that handled processing including
subruns as needed for \METAPOST. This is why, years later, adding a \METAPOST\
library to \LUATEX\ was a logical step. As \CONTEXT\ came with a lot of scripts
(for all kind of tasks related to typesetting and managing a \TEX\ ecosystem)
adding a scripting language (like \LUA) was not that strange either.

In parallel to \PDFTEX\ the experimental \OMEGA\ program was on its way and
although at some point a stable \ALEPH\ variant was there, it never was robust
enough for production. Its main contribution (that survived) was the introduction
if directional typesetting. There were \CONTEXT\ users using it but for very
specific applications. It's also why the bidirectional model of \OMEGA\ inspired
\LUATEX\ more than the simpler model that \ETEX\ used.

\stopsection

\startsection[title={The merge}]

We now move forward to \LUATEX\ and more precisely \LUAMETATEX\ because that is
for \CONTEXT\ the engine of choice now. To what extend is it \TEX\ or not? The
naive answer is \quotation {no} because some primitives are not present and|/|or
are implemented using \LUA. However, these primitives fall into categories. Some
relate to the backend and in \LUAMETATEX\ the backend is not built|-|in and as a
consequence a macro package has to provide the primitives as part of its
implementation of a backend. This is no big deal because the backend related
primitives in \TEX\ The Program are actually examples of extensions and
implemented as such. Handling them happens in kind of isolated code. Take \type
{\special}: it is basically a no|-|op when the \DVI\ driver doesn't interpret
what is passed to the \DVI\ file. \footnote {\CONTEXT\ \MKII\ has a bunch of
backend drivers, \TEX\ code, that targets specific postprocessors and they hook
into primitives like \type {\special} or the additional \type {\pdf...} ones in
\PDFTEX.} \footnote {We need to keep in mind that by the time \PDFTEX\ and later
\LUATEX\ were developed memory constraints were lifted so these engines didn't
have to work around the limitations that for instance \ETEX\ and \OMEGA\ had to
cope with.}

A more drastic change is the lack of font loaders and that no fonts can be stored
in the format. Again this relates to the simple fact that todays fonts are more
demanding so we need to extend the machinery and as we do that via \LUA\
extensions we can as well do all that way. Less drastic, but it could have side
effects, is that the machinery has to be able to deal with \OPENTYPE\ math. And
of course all is \UNICODE\ aware so additional primitives cope with that. But in
principle the old stuff should still work. Hyphenation is also expanded: patterns
are loaded runtime and the hyphenation, ligature building and kerning stages are
split, which actually it a good thing.

The \LUAMETATEX\ code base is a follow up on \LUATEX, that combines good old
\TEX\ (but adapted with respect to fonts, languages and math as mentioned), parts
of \ETEX\ (so it provides more primitives), bits of \PDFTEX\ (like protrusion and
expansion, although adapted), and rudiments of \OMEGA\ (\ALEPH). And of course
there's a lot of new stuff too, primitives as well as ways to plug in \LUA\ code
plus some helpers at the \LUA\ end.

As an example of progression, by now the \ETEX\ extensions that we kept are
integrated more naturally in existing subsystems. A nice detail is that there are
no longer any version numbers that relate to \ETEX; for a while they were kept
but suddenly I realized that it makes no sense to waste (four) command codes on
something that is of not much use: there has never been a real \ETEX\ follow up
after its stable release so testing for a version makes no sense. No backend
means no \PDFTEX\ version info too and \OMEGA\ version numbers serve no purpose
either. If a macro package needs to know what functionality is there, testing for
the \LUATEX\ version number, revision and maybe functionality level makes enough
sense. By the way, one reason for a clean up related to \ETEX\ was that where
\ETEX\ uses change files to replace or extend good old \TEX\ code, \LUATEX\ has
one integrated code base.

\stopsection

\startsection[title={The verdict}]

So in the end the answer is that \LUAMETATEX\ is mostly \TEX\ but that due to
developments like for instance \UNICODE, \OPENTYPE\ fonts and math, as well as
the wish to use images, color, runtime graphics, directionality, features beyond
what the engine has built, etc.\ in the end it hopefully meets the demands to
today. In its core the same code is still there although extensions and hooks got
mixed in more naturally. When in documents (or talks) I speak of \TEX\ I
basically refer to a concept (materialized in the set of core primitives and
related functionality) but once extensions come into play I try to talk of
\LUATEX\ or \LUAMETATEX. This happens kind of automatic because I know what got
added but I can imagine that users who entered the game later don't always see
what was added (and when).

\stopsection

\stopchapter

\stopcomponent

% Written on the day I heard Neal Peart had passed away so mixed with watching
% Rush dvd's which made me realize that times flies. They day before I was at a
% Jan Akkerman show ... timeless quality too.
