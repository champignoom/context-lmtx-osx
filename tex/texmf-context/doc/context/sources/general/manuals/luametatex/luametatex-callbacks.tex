% language=us runpath=texruns:manuals/luametatex

\environment luametatex-style

\startcomponent luametatex-callbacks

\startchapter[reference=callbacks,title={\LUA\ callbacks}]

\startsection[title={Registering callbacks}][library=callback]

\topicindex{callbacks}

\libindex{register}
\libindex{list}
\libindex{find}
\libindex{known}

{\em The callbacks are a moving target. Don't bother me with questions about
them. Some are new and/or experimental and therefore not yet documented. In
\CONTEXT\ we can easily adapt interfaces so changes in these have no real effect
on users. Of course in due time all will be official and documented.}

This library has functions that register, find and list callbacks. Callbacks are
\LUA\ functions that are called in well defined places. There are two kinds of
callbacks: those that mix with existing functionality, and those that (when
enabled) replace functionality. In most cases the second category is expected to
behave similar to the built in functionality because in a next step specific data
is expected. For instance, you can replace the hyphenation routine. The function
gets a list that can be hyphenated (or not). The final list should be valid and
is (normally) used for constructing a paragraph. Another function can replace the
ligature builder and|/|or kern routine. Doing something else is possible but in
the end might not give the user the expected outcome.

The first thing you need to do is registering a callback:

\startfunctioncall
id = callback.register(<string> callback_name, <function> func)
id = callback.register(<string> callback_name, nil)
id = callback.register(<string> callback_name, false)
\stopfunctioncall

Here the \syntax {callback_name} is a predefined callback name, see below. The
function returns the internal \type {id} of the callback or \type {nil}, if the
callback could not be registered.

\LUATEX\ internalizes the callback function in such a way that it does not matter
if you redefine a function accidentally.

Callback assignments are always global. You can use the special value \type {nil}
instead of a function for clearing the callback.

For some minor speed gain, you can assign the boolean \type {false} to the
non|-|file related callbacks, doing so will prevent \LUATEX\ from executing
whatever it would execute by default (when no callback function is registered at
all). Be warned: this may cause all sorts of grief unless you know \notabene
{exactly} what you are doing!

\startfunctioncall
<table> info =
    callback.list()
\stopfunctioncall

The keys in the table are the known callback names, the value is a boolean where
\type {true} means that the callback is currently set (active).

\startfunctioncall
<function> f = callback.find(callback_name)
\stopfunctioncall

If the callback is not set, \type {find} returns \type {nil}. The \type {known}
function can be used to check if a callback is supported.

\startfunctioncall
if callback.known("foo") then ... end
\stopfunctioncall

\stopsection

\startsection[title={File related callbacks},reference=iocallback][library=callback]

\startsubsection[title={\cbk {find_format_file} and \cbk {find_log_file}}]

\topicindex{callbacks+format file}
\topicindex{callbacks+log file}

These callbacks are called as:

\startfunctioncall
<string> actualname =
    function (<string> askedname)
\stopfunctioncall

The \type {askedname} is a format file for reading (the format file for writing
is always opened in the current directory) or a log file for writing.

\stopsubsection

\startsubsection[title={\cbk {open_data_file}}]

\topicindex{callbacks+opening files}

This callback function gets a filename passed:

\startfunctioncall
<table> env = function (<string> filename)
\stopfunctioncall

The return value is either the boolean value false or a table with two functions.
A mandate \type {reader} function fill be called once for each new line to be
read, the optional \type {close} function will be called once \LUATEX\ is done
with the file.

\LUATEX\ never looks at the rest of the table, so you can use it to store your
private per|-|file data. Both the callback functions will receive the table as
their only argument.

% No longer needed anyway.
%
% \subsection{\cbk {if_end_of_file}}
%
% \topicindex{callbacks+checking files}
%
% This callback has no arguments and your function should return true or false. The
% callback is triggered by \type {\ifeof}. It's up to the macro package to come up
% with a reasonable implementation. By default the test is always true.
%
% \startfunctioncall
% <boolean> eof =
%     function ()
% \stopfunctioncall

\stopsubsection

\stopsection

\startsection[title={Data processing callbacks}][library=callback]

\startsubsection[title={\cbk {process_jobname}}]

\topicindex{callbacks+jobname}

This callback allows you to change the jobname given by \prm {jobname} in \TEX\
and \type {tex.jobname} in Lua. It does not affect the internal job name or the
name of the output or log files.

\startfunctioncall
function(<string> jobname)
    return <string> adjusted_jobname
end
\stopfunctioncall

The only argument is the actual job name; you should not use \type {tex.jobname}
inside this function or infinite recursion may occur. If you return \type {nil},
\LUATEX\ will pretend your callback never happened. This callback does not
replace any internal code.

\stopsubsection

\stopsection

\startsection[title={Node list processing callbacks}][library=callback]

The description of nodes and node lists is in~\in{chapter}[nodes].

\startsubsection[title={\cbk {contribute_filter}}]

\topicindex{callbacks+contributions}

This callback is called when \LUATEX\ adds contents to list:

\startfunctioncall
function(<string> extrainfo)
end
\stopfunctioncall

The string reports the group code. From this you can deduce from
what list you can give a treat.

\starttabulate[|l|p|]
\DB value             \BC explanation                                  \NC \NR
\TB
\NC \type{pre_box}    \NC interline material is being added            \NC \NR
\NC \type{pre_adjust} \NC \prm {vadjust} material is being added       \NC \NR
\NC \type{box}        \NC a typeset box is being added (always called) \NC \NR
\NC \type{adjust}     \NC \prm {vadjust} material is being added       \NC \NR
\LL
\stoptabulate

\stopsubsection

\startsubsection[title={\cbk {buildpage_filter}}]

\topicindex{callbacks+building pages}

This callback is called whenever \LUATEX\ is ready to move stuff to the main
vertical list. You can use this callback to do specialized manipulation of the
page building stage like imposition or column balancing.

\startfunctioncall
function(<string> extrainfo)
end
\stopfunctioncall

The string \type {extrainfo} gives some additional information about what \TEX's
state is with respect to the \quote {current page}. The possible values for the
\cbk {buildpage_filter} callback are:

\starttabulate[|l|p|]
\DB value                  \BC explanation                             \NC \NR
\TB
\NC \type{alignment}       \NC a (partial) alignment is being added    \NC \NR
\NC \type{after_output}    \NC an output routine has just finished     \NC \NR
\NC \type{new_graf}        \NC the beginning of a new paragraph        \NC \NR
\NC \type{vmode_par}       \NC \prm {par} was found in vertical mode   \NC \NR
\NC \type{hmode_par}       \NC \prm {par} was found in horizontal mode \NC \NR
\NC \type{insert}          \NC an insert is added                      \NC \NR
\NC \type{penalty}         \NC a penalty (in vertical mode)            \NC \NR
\NC \type{before_display}  \NC immediately before a display starts     \NC \NR
\NC \type{after_display}   \NC a display is finished                   \NC \NR
\NC \type{end}             \NC \LUATEX\ is terminating (it's all over) \NC \NR
\LL
\stoptabulate

\stopsubsection

\startsubsection[title={\cbk {build_page_insert}}]

\topicindex{callbacks+inserts}

This callback is called when the page builder adds an insert. There is not much
control over this mechanism but this callback permits some last minute
manipulations of the spacing before an insert, something that might be handy when
for instance multiple inserts (types) are appended in a row.

\startfunctioncall
function(<number> n, <number> i)
    return <number> register
end
\stopfunctioncall

with

\starttabulate[|l|p|]
\DB value    \BC explanation             \NC \NR
\TB
\NC \type{n} \NC the insert class        \NC \NR
\NC \type{i} \NC the order of the insert \NC \NR
\LL
\stoptabulate

The return value is a number indicating the skip register to use for the
prepended spacing. This permits for instance a different top space (when \type
{i} equals one) and intermediate space (when \type {i} is larger than one). Of
course you can mess with the insert box but you need to make sure that \LUATEX\
is happy afterwards.

\stopsubsection

\startsubsection[title={\cbk {pre_linebreak_filter}}]

\topicindex{callbacks+linebreaks}

This callback is called just before \LUATEX\ starts converting a list of nodes
into a stack of \prm {hbox}es, after the addition of \prm {parfillskip}.

\startfunctioncall
function(<node> head, <string> groupcode)
    return <node> newhead
end
\stopfunctioncall

The string called \type {groupcode} identifies the nodelist's context within
\TEX's processing. The range of possibilities is given in the table below, but
not all of those can actually appear in \cbk {pre_linebreak_filter}, some are
for the \cbk {hpack_filter} and \cbk {vpack_filter} callbacks that will be
explained in the next two paragraphs.

\starttabulate[|l|p|]
\DB value                \BC explanation                                 \NC \NR
\TB
\NC \type{<empty>}       \NC main vertical list                          \NC \NR
\NC \type{hbox}          \NC \prm {hbox} in horizontal mode              \NC \NR
\NC \type{adjusted_hbox} \NC \prm {hbox} in vertical mode                \NC \NR
\NC \type{vbox}          \NC \prm {vbox}                                 \NC \NR
\NC \type{vtop}          \NC \prm {vtop}                                 \NC \NR
\NC \type{align}         \NC \prm {halign} or \prm {valign}              \NC \NR
\NC \type{disc}          \NC discretionaries                             \NC \NR
\NC \type{insert}        \NC packaging an insert                         \NC \NR
\NC \type{vcenter}       \NC \prm {vcenter}                              \NC \NR
\NC \type{local_box}     \NC \prm {localleftbox} or \prm {localrightbox} \NC \NR
\NC \type{split_off}     \NC top of a \prm {vsplit}                      \NC \NR
\NC \type{split_keep}    \NC remainder of a \prm {vsplit}                \NC \NR
\NC \type{align_set}     \NC alignment cell                              \NC \NR
\NC \type{fin_row}       \NC alignment row                               \NC \NR
\LL
\stoptabulate

As for all the callbacks that deal with nodes, the return value can be one of
three things:

\startitemize
\startitem
    boolean \type {true} signals successful processing
\stopitem
\startitem
    \type {<node>} signals that the \quote {head} node should be replaced by the
    returned node
\stopitem
\startitem
    boolean \type {false} signals that the \quote {head} node list should be
    ignored and flushed from memory
\stopitem
\stopitemize

This callback does not replace any internal code.

\stopsubsection

\startsubsection[title={\cbk {linebreak_filter}}]

\topicindex{callbacks+linebreaks}

This callback replaces \LUATEX's line breaking algorithm.

\startfunctioncall
function(<node> head, <boolean> is_display)
    return <node> newhead
end
\stopfunctioncall

The returned node is the head of the list that will be added to the main vertical
list, the boolean argument is true if this paragraph is interrupted by a
following math display.

If you return something that is not a \type {<node>}, \LUATEX\ will apply the
internal linebreak algorithm on the list that starts at \type {<head>}.
Otherwise, the \type {<node>} you return is supposed to be the head of a list of
nodes that are all allowed in vertical mode, and at least one of those has to
represent an \prm {hbox}. Failure to do so will result in a fatal error.

Setting this callback to \type {false} is possible, but dangerous, because it is
possible you will end up in an unfixable \quote {deadcycles loop}.

\stopsubsection

\startsubsection[title={\type {append_to_vlist_filter}}]

\topicindex{callbacks+contributions}

This callback is called whenever \LUATEX\ adds a box to a vertical list (the
\type {mirrored} argument is obsolete):

\startfunctioncall
function(<node> box, <string> locationcode, <number> prevdepth)
    return list [, prevdepth [, checkdepth ] ]
end
\stopfunctioncall

It is ok to return nothing or \type {nil} in which case you also need to flush
the box or deal with it yourself. The prevdepth is also optional. Locations are
\type {box}, \type {alignment}, \type {equation}, \type {equation_number} and
\type {post_linebreak}. When the third argument returned is \type {true} the
normal prevdepth correction will be applied, based on the first node.

\stopsubsection

\startsubsection[title={\cbk {post_linebreak_filter}}]

\topicindex{callbacks+linebreaks}

This callback is called just after \LUATEX\ has converted a list of nodes into a
stack of \prm {hbox}es.

\startfunctioncall
function(<node> head, <string> groupcode)
    return <node> newhead
end
\stopfunctioncall

This callback does not replace any internal code.

\stopsubsection

\startsubsection[title={\cbk {glyph_run}}]

\topicindex{callbacks+fonts}
\topicindex{callbacks+hyphenation}
\topicindex{callbacks+kerning}
\topicindex{callbacks+ligature building}

When set this callback is triggered when \TEX\ normally handles the ligaturing
and kerning. In \LUATEX\ you use the \typ {hpack_filter} and \typ
{per_linebreak_filter} callbacks for that (where each passes different
arguments). This callback doesn't get triggered when there are no glyphs (in
\LUATEX\ this optimization is controlled by a a variable).

\startfunctioncall
function(<node> head, <string> groupcode, <number> direction])
    return <node> newhead
end
\stopfunctioncall

The traditional \TEX\ font processing is bypassed so you need to take care of that
with the helpers. (For the moment we keep the ligaturing and kerning callbacks but
they are kind of obsolete.)

\stopsubsection

\startsubsection[title={\cbk {hpack_filter}}]

\topicindex{callbacks+packing}

This callback is called when \TEX\ is ready to start boxing some horizontal mode
material. Math items and line boxes are ignored at the moment.

\startfunctioncall
function(<node> head, <string> groupcode, <number> size,
         <string> packtype [, <number> direction] [, <node> attributelist])
    return <node> newhead
end
\stopfunctioncall

The \type {packtype} is either \type {additional} or \type {exactly}. If \type
{additional}, then the \type {size} is a \type {\hbox spread ...} argument. If
\type {exactly}, then the \type {size} is a \type {\hbox to ...}. In both cases,
the number is in scaled points.

This callback does not replace any internal code.

\stopsubsection

\startsubsection[title={\cbk {vpack_filter}}]

\topicindex{callbacks+packing}

This callback is called when \TEX\ is ready to start boxing some vertical mode
material. Math displays are ignored at the moment.

This function is very similar to the \cbk {hpack_filter}. Besides the fact
that it is called at different moments, there is an extra variable that matches
\TEX's \prm {maxdepth} setting.

\startfunctioncall
function(<node> head, <string> groupcode, <number> size, <string> packtype,
        <number> maxdepth [, <number> direction] [, <node> attributelist]))
    return <node> newhead
end
\stopfunctioncall

This callback does not replace any internal code.

\stopsubsection

\startsubsection[title={\cbk {packed_vbox_filter}}]

\topicindex{callbacks+packing}

After the \cbk {vpack_filter} callback (see previous section) is triggered the
box get packed and after that this callback can be configured to kick in.

\startfunctioncall
function(<node> head, <string> groupcode)
    return <node> newhead
end
\stopfunctioncall

This callback does not replace any internal code.

\stopsubsection

\startsubsection[title={\cbk {alignment_filter}}]

\topicindex{callbacks+alignments}

This is an experimental callback that when set is called several times during the
construction of an alignment. The context values are available in \typ
{tex.getalignmentcontextvalues()}.

\startfunctioncall
function(<node> head, <string> context, <node> attributes, <node> preamble)
    -- no return values
end
\stopfunctioncall

There are no sanity checks so if a user messes up the passed node lists the results
can be unpredictable and, as with other node related callbacks, crash the engine.

\stopsubsection

\startsubsection[title={\cbk {localbox_filter}}]

\topicindex{callbacks+local boxes}

Local boxes are a somewhat tricky and error prone feature so use this callback
with care because the paragraph is easily messed up. A line can have a left,
right and middle box where the middle one has no width. The callback gets quite
some parameters passed:

\startfunctioncall
function(<node> linebox, <node> leftbox, <node> rightbox, <node> middlebox,
    <number> linenumber,
    <number> leftskip, <number> rightskip, <number> lefthang, <number> righthang,
    <number> indentation, <number> parinitleftskip, <number> parinitrightskip,
    <number> parfillleftskip, <number> parfillrightskip,
    <number> overshoot)
    -- no return values
end
\stopfunctioncall

This is an experimental callback that will be tested in different \CONTEXT\
mechanisms before it will be declared stable.

\stopsubsection

\startsubsection[title={\cbk {process_rule}}]

\topicindex{callbacks+rules}

This is an experimental callback. It can be used with rules of subtype~4
(user). The callback gets three arguments: the node, the width and the
height. The callback can use \type {pdf.print} to write code to the \PDF\
file but beware of not messing up the final result. No checking is done.

\stopsubsection

\startsubsection[title={\type {pre_output_filter}}]

\topicindex{callbacks+output}

This callback is called when \TEX\ is ready to start boxing the box 255 for \prm
{output}.

\startfunctioncall
function(<node> head, <string> groupcode, <number> size, <string> packtype,
        <number> maxdepth [, <number> direction])
    return <node> newhead
end
\stopfunctioncall

This callback does not replace any internal code.

\stopsubsection

\startsubsection[title={\cbk {hyphenate}}]

\topicindex{callbacks+hyphenation}

This callback is supposed to insert discretionary nodes in the node list it
receives.

\startfunctioncall
function(<node> head, <node> tail)
    -- no return values
end
\stopfunctioncall

Setting this callback to \type {false} will prevent the internal discretionary
insertion pass.

\stopsubsection

\startsubsection[title={\cbk {ligaturing}}]

\topicindex{callbacks+ligature building}

This callback, which expects no return values, has to apply ligaturing to the
node list it receives.

\startfunctioncall
function(<node> head, <node> tail)
    -- no return values
end
\stopfunctioncall

You don't have to worry about return values because the \type {head} node that is
passed on to the callback is guaranteed not to be a glyph_node (if need be, a
temporary node will be prepended), and therefore it cannot be affected by the
mutations that take place. After the callback, the internal value of the \quote
{tail of the list} will be recalculated.

The \type {next} of \type {head} is guaranteed to be non-nil. The \type {next} of
\type {tail} is guaranteed to be nil, and therefore the second callback argument
can often be ignored. It is provided for orthogonality, and because it can
sometimes be handy when special processing has to take place.

Setting this callback to \type {false} will prevent the internal ligature
creation pass. You must not ruin the node list. For instance, the head normally
is a local par node, and the tail a glue. Messing too much can push \LUATEX\ into
panic mode.

\stopsubsection

\startsubsection[title={\cbk {kerning}}]

\topicindex{callbacks+kerning}

This callback has to apply kerning between the nodes in the node list it
receives. See \cbk {ligaturing} for calling conventions.

\startfunctioncall
function(<node> head, <node> tail)
    -- no return values
end
\stopfunctioncall

Setting this callback to \type {false} will prevent the internal kern insertion
pass. You must not ruin the node list. For instance, the head normally is a local
par node, and the tail a glue. Messing too much can push \LUATEX\ into panic
mode.

\stopsubsection

\startsubsection[title={\cbk {append_line_filter}}]

\topicindex{callbacks+lines}

Every time a line is added this callback is triggered, when set. migrated
material and adjusts also qualify as such and the detail relates to the adjust
index.

\startfunctioncall
function(<node> head, <node> tail, <string> context, <number> detail)
    return <node> newhead
end
\stopfunctioncall

A list of possible context values can be queried with \typ
{tex.getappendlinecontextvalues()}.

\stopsubsection

\stopsection

\startsection[title={Paragraph callbacks}]

\startsubsection[title={\type {insert_par}}]

\topicindex{callbacks+paragraphs}

Each paragraph starts with a local par node that keeps track of for instance
the direction. You can hook a callback into the creator:

\startfunctioncall
function(<node> par, <string> location)
    -- no return values
end
\stopfunctioncall

There is no return value and you should make sure that the node stays valid
as otherwise \TEX\ can get confused.

\stopsubsection

\startsubsection[title={\cbk {begin_paragraph}}]

\topicindex{callbacks+paragraphs}

{\em todo}

\stopsubsection

\startsubsection[title={\cbk {paragraph_context}}]

\topicindex{callbacks+paragraphs}

{\em todo}

\stopsubsection

\startsection[title={Math related callbacks}]

\startsubsection[title={\cbk {mlist_to_hlist}}]

\topicindex{callbacks+math}

This callback replaces \LUATEX's math list to node list conversion algorithm.

\startfunctioncall
function(<node> head, <string> display_type, <boolean> need_penalties)
    return <node> newhead
end
\stopfunctioncall

The returned node is the head of the list that will be added to the vertical or
horizontal list, the string argument is either \quote {text} or \quote {display}
depending on the current math mode, the boolean argument is \type {true} if
penalties have to be inserted in this list, \type {false} otherwise.

Setting this callback to \type {false} is bad, it will almost certainly result in
an endless loop.

\stopsubsection

\startsubsection[title={\cbk {math_rule}}]

\topicindex{callbacks+math}
\topicindex{callbacks+ruled}

{\em todo}

\stopsubsection

\startsubsection[title={\cbk {make_extensible}}]

\topicindex{callbacks+math}
\topicindex{callbacks+fonts}

{\em todo}

\stopsubsection

\startsubsection[title={\cbk {register_extensible}}]

\topicindex{callbacks+math}
\topicindex{callbacks+fonts}

{\em todo}

\stopsubsection

\stopsection

\startsection[title={Information reporting callbacks}][library=callback]

\startsubsection[title={\cbk {pre_dump}}]

\topicindex{callbacks+dump}

\startfunctioncall
function()
    -- no return values
end
\stopfunctioncall

This function is called just before dumping to a format file starts. It does not
replace any code and there are neither arguments nor return values.

\stopsubsection

\startsubsection[title={\cbk {start_run}}]

\topicindex{callbacks+job run}

\startfunctioncall
function()
    -- no return values
end
\stopfunctioncall

This callback replaces the code that prints \LUATEX's banner. Note that for
successful use, this callback has to be set in the \LUA\ initialization script,
otherwise it will be seen only after the run has already started.

\stopsubsection

\startsubsection[title={\cbk {stop_run}}]

\topicindex{callbacks+job run}

\startfunctioncall
function()
end
\stopfunctioncall

This callback replaces the code that prints \LUATEX's statistics and \quote
{output written to} messages. The engine can still do housekeeping and therefore
you should not rely on this hook for postprocessing the \PDF\ or log file.

\stopsubsection

\startsubsection[title={\cbk {intercept_tex_error}, \cbk {intercept_lua_error}}]

\topicindex{callbacks+errors}

\startfunctioncall
function()
    -- no return values
end
\stopfunctioncall

This callback is run from inside the \TEX\ error function, and the idea is to
allow you to do some extra reporting on top of what \TEX\ already does (none of
the normal actions are removed). You may find some of the values in the \type
{status} table useful. The \TEX\ related callback gets two arguments: the current
processing mode and a boolean indicating if there was a runaway.

\stopsubsection

\startsubsection[title={\cbk {show_error_message} and \cbk {show_warning_message}}]

\topicindex{callbacks+errors}
\topicindex{callbacks+warnings}

\startfunctioncall
function()
    -- no return values
end
\stopfunctioncall

These callback replaces the code that prints the error message. The usual
interaction after the message is not affected.

\stopsubsection

\startsubsection[title={\cbk {start_file}}]

\topicindex{callbacks+files}

\startfunctioncall
function(category,filename)
    -- no return values
end
\stopfunctioncall

This callback replaces the code that \LUATEX\ prints when a file is opened like
\type {(filename} for regular files. The category is a number:

\starttabulate[|c|l|]
\DB value  \BC meaning \NC \NR
\TB
\NC 1 \NC a normal data file, like a \TEX\ source \NC \NR
\NC 2 \NC a font map coupling font names to resources \NC \NR
\NC 3 \NC an image file (\type {png}, \type {pdf}, etc) \NC \NR
\NC 4 \NC an embedded font subset \NC \NR
\NC 5 \NC a fully embedded font \NC \NR
\LL
\stoptabulate

\stopsubsection

\startsubsection[title={\cbk {stop_file}}]

\topicindex{callbacks+files}

\startfunctioncall
function(category)
    -- no return values
end
\stopfunctioncall

This callback replaces the code that \LUATEX\ prints when a file is closed like
the \type {)} for regular files.

\stopsubsection

\startsubsection[title={\cbk {wrapup_run}}]

\topicindex{callbacks+wrapping up}

This callback is called after the \PDF\ and log files are closed. Use it at your own
risk.

\stopsubsection

\stopsection

\startsection[title={Font-related callbacks}][library=callback]

\startsubsection[title={\cbk {define_font}}]

\topicindex{callbacks+fonts}

\startfunctioncall
function(<string> name, <number> size)
    return <number> id
end
\stopfunctioncall

The string \type {name} is the filename part of the font specification, as given
by the user.

The number \type {size} is a bit special:

\startitemize[packed]
\startitem
    If it is positive, it specifies an \quote{at size} in scaled points.
\stopitem
\startitem
    If it is negative, its absolute value represents a \quote {scaled} setting
    relative to the design size of the font.
\stopitem
\stopitemize

The font can be defined with \type {font.define} which returns a font identifier
that can be returned in the callback. So, contrary to \LUATEX, in \LUAMETATEX\
we only accept a number.

The internal structure of the \type {font} table that is passed to \type
{font.define} is explained in \in {chapter} [fonts]. That table is saved
internally, so you can put extra fields in the table for your later \LUA\ code to
use. In alternative, \type {retval} can be a previously defined fontid. This is
useful if a previous definition can be reused instead of creating a whole new
font structure.

Setting this callback to \type {false} is pointless as it will prevent font
loading completely but will nevertheless generate errors.

\stopsubsection

\startsubsection[title={\cbk {missing_character} and \cbk {process_character}}]

\topicindex{callbacks+fonts}
\topicindex{callbacks+characters}

This callback is triggered when a character node is created and the font doesn't
have the requested character.

\startfunctioncall
function(<node> glyph, <number> font, <number> character)
    -- no return value
end
\stopfunctioncall

The \type {process_character} callback is experimental and gets called when a
glyph node is created and the callback field in a character is set.

\startfunctioncall
function(<number> font, <number> character)
    -- no return value
end
\stopfunctioncall

\stopsubsection

\stopsection

\startsection[title=Reporting]

\startsubsection[title={\cbk {show_whatsit}}]

\topicindex{callbacks+whatsits}

Because we only have a generic whatsit it is up to the macro package to provide
details when tracing them.

\startfunctioncall
function(<node> whatsit, <number> indentation,
    <number> tracinglevel, <number> currentlevel, <number> inputlevel)
    -- no return value
end
\stopfunctioncall

The indentation tells how many periods are to be typeset if you want to be
compatible with the rest of tracing. The tracinglevels indicates if the current
level and\|/or input level are shown cf. \prm {tracinglevels}. Of course one
is free to show whatever in whatever way suits the whatsit best.

\stopsubsection

\startsubsection[title={\cbk {get_attribute}}]

\topicindex{callbacks+attributes}

Because attributes are abstract pairs of indices and values the reported
properties makes not much sense and are very macro package (and user) dependent.
This callback permits more verbose reporting by the engine when tracing is
enabled.

\startfunctioncall
function(<number> index, <number> value)
    return <string>, <string>
end
\stopfunctioncall

\stopsubsection

\startsubsection[title={\cbk {get_noad_class}}]

\topicindex{callbacks+classes}

We have built|-|in math classes but there can also be user defined ones. This
callback can be used to report more meaningful strings instead of numbers when
tracing.

\startfunctioncall
function(<number> class)
    return <string>
end
\stopfunctioncall

\stopsubsection

\startsubsection[title={\cbk {trace_memory}}]

When the engine starts all kind of memory is pre|-|allocated> depending on the
configuration more gets allocated when a category runs out of memory. The
\LUAMETATEX\ engine is more dynamic than \LUATEX. If this callback is set it will
get called as follows:

\startfunctioncall
function(<string> category, <boolean> success)
    -- no return value
end
\stopfunctioncall

The boolean indicates if the allocation has been successful. One can best quit
the run when this one is \type {false}, if the engine doesn't already do that.

\stopsubsection

\startsubsection[title={\type {hpack_quality}}]

\topicindex{callbacks+packing}

This callback can be used to intercept the overfull messages that can result from
packing a horizontal list (as happens in the par builder). The function takes a
few arguments:

\startfunctioncall
function(<string> incident, <number> detail, <node> head, <number> first,
         <number> last)
    return <node> whatever
end
\stopfunctioncall

The incident is one of \type {overfull}, \type {underfull}, \type {loose} or
\type {tight}. The detail is either the amount of overflow in case of \type
{overfull}, or the badness otherwise. The head is the list that is constructed
(when protrusion or expansion is enabled, this is an intermediate list).
Optionally you can return a node, for instance an overfull rule indicator. That
node will be appended to the list (just like \TEX's own rule would).

\stopsubsection

\startsubsection[title={\type {vpack_quality}}]

\topicindex{callbacks+packing}

This callback can be used to intercept the overfull messages that can result from
packing a vertical list (as happens in the page builder). The function takes a
few arguments:

\startfunctioncall
function(<string> incident, <number> detail, <node> head, <number> first,
    <number> last)
end
\stopfunctioncall

The incident is one of \type {overfull}, \type {underfull}, \type {loose} or
\type {tight}. The detail is either the amount of overflow in case of \type
{overfull}, or the badness otherwise. The head is the list that is constructed.

\stopsubsection

\startsubsection[title={\type {show_lua_call}}]

\topicindex{callbacks+lua}

This one can be used to help reporting definitions that relate to \LUA\ calls to
be more meaningful when tracing.

\startfunctioncall
function(<string> name, <number> index)
    return <string>
end
\stopfunctioncall

\stopsubsection

\startsubsection[title={\type {handle_overload}}]

\topicindex{callbacks+overload}

This is one of the few callbacks that is aimed at \CONTEXT: it relates to overload
protection of macros and other variables.

\startfunctioncall
function(<boolean> error, <number> overload, <string> csname, <number> flags)
    -- no return values
end
\stopfunctioncall

The overload is determined by:

\starttabulate[|c|l|c|c|c|c|c|]
\DB   \BC         \BC immutable \BC permanent \BC primitive \BC frozen \BC instance \NC \NR
\TB
\NC 1 \NC warning \NC \star     \NC \star     \NC \star     \NC        \NC          \NC \NR
\NC 2 \NC error   \NC \star     \NC \star     \NC \star     \NC        \NC          \NC \NR
\NC 3 \NC warning \NC \star     \NC \star     \NC \star     \NC \star  \NC          \NC \NR
\NC 4 \NC error   \NC \star     \NC \star     \NC \star     \NC \star  \NC          \NC \NR
\NC 5 \NC warning \NC \star     \NC \star     \NC \star     \NC \star  \NC \star    \NC \NR
\NC 6 \NC error   \NC \star     \NC \star     \NC \star     \NC \star  \NC \star    \NC \NR
\LL
\stoptabulate

This relates to the optional prefixed that can be used when defining and setting
quantities and is therefore also a bit of a playground. All macros and aliases in
\CONTEXT\ are classified this way.

\stopsection

\stopchapter

\stopcomponent
