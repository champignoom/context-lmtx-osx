% language=us

\usemodule[nodechart]

\startcomponent about-nodes

\environment about-environment

\startchapter[title={Juggling nodes}]

\startsection[title=Introduction]

When you use \TEX, join the community, follow mailing lists, read manuals,
and|/|or attend meetings, there will come a moment when you run into the word
\quote {node}. But, as a regular user, even if you write macros, you can happily
ignore them because in practice you will never really see them. They are hidden
deep down in \TEX.

Some expert \TEX ies love to talk about \TEX's mouth, stomach, gut and other
presumed bodily elements. Maybe it is seen as proof of the deeper understanding
of this program as Don Knuth uses these analogies in his books about \TEX\ when
he discusses how \TEX\ reads the input, translates it and digests it into a
something that can be printed or viewed. No matter how your input gets digested,
at some point we get nodes. However, as users have no real access to the
internals, nodes never show themselves to the user. They have no bodily analogy
either.

A character that is read from the input can become a character node. Multiple
characters can become a linked list of nodes. Such a list can contain other kind
of nodes as well, for instance spaced become glue. There can also be penalties
that steer the machinery. And kerns too: fixed displacements. Such a list can be
wrapped in a box. In the process hyphenation is applied, characters become glyphs
and intermediate math nodes becomes a combination of regular glyphs, kerns and
glue, wrapped into boxes. So, an hbox that contains the three glyphs \type {tex}
can be represented as follows:

\startlinecorrection
    \setupFLOWchart
      [dx=2em,
       dy=1em,
       width=4em,
       height=2em]
    \setupFLOWshapes
      [framecolor=maincolor]
    \startFLOWchart[nodes]
      \startFLOWcell
        \name       {box}
        \location   {1,1}
        \shape      {action}
        \text       {hbox}
        \connection [rl] {t}
      \stopFLOWcell
      \startFLOWcell
        \name       {t}
        \location   {2,1}
        \shape      {action}
        \text       {t}
        \connection [+t-t] {e}
      \stopFLOWcell
      \startFLOWcell
        \name       {e}
        \location   {3,1}
        \shape      {action}
        \text       {e}
        \connection [+t-t] {x}
        \connection [-b+b] {t}
      \stopFLOWcell
      \startFLOWcell
        \name       {x}
        \location   {4,1}
        \shape      {action}
        \text       {x}
        \connection [-b+b] {e}
      \stopFLOWcell
    \stopFLOWchart
    \FLOWchart[nodes]
\stoplinecorrection

Eventually a long sequence of nodes can become a paragraph of lines and each line
is a box. The lines together make a page which is also a box. There are many kind
of nodes but some are rather special and don't translate directly to some visible
result. When dealing with \TEX\ as user we can forget about nodes: we never really
see them.

In this example we see an hlist (hbox) node. Such a node has properties like
width, height, depth, shift etc. The characters become glyph nodes that have
(among other properties) a reference to a font, character, language.

Because \TEX\ is also about math, and because math is somewhat special, we have
noads, some intermediate kind of node that makes up a math list, that eventually
gets transformed into a list of nodes. And, as proof of extensibility, Knuth came
up with a special node that is more or less ignored by the machinery but travels
with the list and can be dealt with in special backend code. Their name indicates
what it's about: they are called whatsits (which sounds better that whatevers).
In \LUATEX\ some whatsits are used in the frontend, for instance directional
information is stored in whatsits.

The \LUATEX\ engine not only opens up the \UNICODE\ and \OPENTYPE\ universes, but
also the traditional \TEX\ engine. It gives us access to nodes. And this permits
us to go beyond what was possible before and therefore on mailing lists like the
\CONTEXT\ list, the word node will pop up more frequently. If you look into the
\LUA\ files that ship with \CONTEXT\ you cannot avoid seeing them. And, when you
use the \CLD\ interface you might even want to manipulate them. A nice side
effect is that you can sound like an expert without having to refer to bodily
aspects of \TEX: you just see them as some kind of \LUA\ userdata variable. And
you access them like tables: they are abstracts units with properties.

\stopsection

\startsection[title=Basics]

Nodes are kind of special in the sense that you need to keep an eye on creation
and destruction. In \TEX\ itself this is mostly hidden:

\startbuffer
\setbox0\hbox{some text}
\stopbuffer

\typebuffer

If we look {\em into} this box we get a list of glyphs (see \in {figure}
[fig:dummy:1]).

\startplacefigure[reference=fig:dummy:1]
    \getbuffer
    \boxtoFLOWchart[dummy]{0}
    \small
    \FLOWchart[dummy][width=14em,height=3em,dx=1em,dy=.75em] % ,hcompact=yes]
\stopplacefigure

In \TEX\ you can flush such a box using \type {\box0} or copy it using \type
{\copy0}. You can also flush the contents i.e.\ omit the wrapper using \type
{\unhbox0} and \type {\unhcopy0}. The possibilities for disassembling the
content of a box (or any list for that matter) are limited. In practice you
can consider disassembling to be absent.

This is different at the \LUA\ end: there we can really start at the beginning of
a list, loop over it and see what's in there as well as change, add and remove
nodes. The magic starts with:

\starttyping
local box = tex.box[0]
\stoptyping

Now we have a variable that has a so called \type {hlist} node. This node has not
only properties like \type {width}, \type {height}, \type {depth} and \type
{shift}, but also a pointer to the content: \type {list}.

\starttyping
local list = box.list
\stoptyping

Now, when we start messing with this list, we need to keep into account that the
nodes are in fact userdata objects, that is: they are efficient \TEX\ data
structures that have a \LUA\ interface. At the \TEX\ end the repertoire of
commands that we can use to flush boxes is rather limited and as we cannot mess
with the content we have no memory management issues. However, at the \LUA\ end
this is different. Nodes can have pointers to other nodes and they can even have
special properties that relate to other resources in the program.

Take this example:

\starttyping
\setbox0\hbox{some text}
\directlua{node.write(tex.box[0])}
\stoptyping

At the \TEX\ end we wrap something in a box. Then we can at the \LUA\ end access
that box and print it back into the input. However, as \TEX\ is no longer in
control it cannot know that we already flushed the list. Keep in mind that this
is a simple example, but imagine more complex content, that contains hyperlinks
or so. Now take this:

\starttyping
\setbox0\hbox{some text 1}
\setbox0\hbox{some text 2}
\stoptyping

Here \TEX\ knows that the box has content and it will free the memory beforehand
and forget the first text. Or this:

\starttyping
\setbox0\hbox{some text}
\box0 \box0
\stoptyping

The box will be used and after that it's empty so the second flush is basically a
harmless null operation: nothing gets inserted. But this:

\starttyping
\setbox0\hbox{some text}
\directlua{node.write(tex.box[0])}
\directlua{node.write(tex.box[0])}
\stoptyping

will definitely fail. The first call flushes the box and the second one sees
no box content and will bark. The best solution is to use a copy:

\starttyping
\setbox0\hbox{some text}
\directlua{node.write(node.copy_list(tex.box[0]))}
\stoptyping

That way \TEX\ doesn't see a change in the box and will free it when needed: when
it gets flushed, reassigned, at the end of a group, wherever.

In \CONTEXT\ a somewhat shorter way of printing back to \TEX\ is the following
and we will use that:

\starttyping
\setbox0\hbox{some text}
\ctxlua{context(node.copy_list(tex.box[0])}
\stoptyping

or shortcut into \CONTEXT:

\starttyping
\setbox0\hbox{some text}
\cldcontext{node.copy_list(tex.box[0])}
\stoptyping

As we've now arrived at the \LUA\ end, we have more possibilities with nodes. In
the next sections we will explore some of these.

\stopsection

\startsection[title=Management]

The most important thing to keep in mind is that each node is unique in the sense
that it can be used only once. If you don't need it and don't flush it, you
should free it. If you need it more than once, you need to make a copy. But let's
first start with creating a node.

\starttyping
local g = node.new("glyph")
\stoptyping

This node has some properties that need to be set. The most important are the font
and the character. You can find more in the \LUATEX\ manual.

\starttyping
g.font = font.current()
g.char = utf.byte("a")
\stoptyping

After this we can write it to the \TEX\ input:

\starttyping
context(g)
\stoptyping

This node is automatically freed afterwards. As we're talking \LUA\ you can use
all kind of commands that are defined in \CONTEXT. Take fonts:

\startbuffer
\startluacode
local g1 = node.new("glyph")
local g2 = node.new("glyph")

g1.font = fonts.definers.internal {
    name = "dejavuserif",
    size = "60pt",
}

g2.font = fonts.definers.internal {
    name = "dejavusansmono",
    size = "60pt",
}

g1.char = utf.byte("a")
g2.char = utf.byte("a")

context(g1)
context(g2)
\stopluacode
\stopbuffer

\typebuffer

We get: \getbuffer, but there is one pitfall: the nodes have to be flushed in
horizontal mode, so either put \type {\dontleavehmode} in front or add \type
{context.dontleavehmode()}. If you get error messages like \typ {this can't
happen} you probably forgot to enter horizontal mode.

In \CONTEXT\ you have some helpers, for instance:

\starttyping
\startluacode
local id = fonts.definers.internal { name = "dejavuserif" }

context(nodes.pool.glyph(id,utf.byte("a")))
context(nodes.pool.glyph(id,utf.byte("b")))
context(nodes.pool.glyph(id,utf.byte("c")))
\stopluacode
\stoptyping

or, when we need these functions a lot and want to save some typing:

\startbuffer
\startluacode
local getfont  = fonts.definers.internal
local newglyph = nodes.pool.glyph
local utfbyte  = utf.byte

local id = getfont { name = "dejavuserif" }

context(newglyph(id,utfbyte("a")))
context(newglyph(id,utfbyte("b")))
context(newglyph(id,utfbyte("c")))
\stopluacode
\stopbuffer

\typebuffer

This renders as: \getbuffer. We can make copies of nodes too:

\startbuffer
\startluacode
local id = fonts.definers.internal { name = "dejavuserif" }
local a  = nodes.pool.glyph(id,utf.byte("a"))

for i=1,10 do
    context(node.copy(a))
end

node.free(a)
\stopluacode
\stopbuffer

\typebuffer

This gives: \getbuffer. Watch how afterwards we free the node. If we have not one
node but a list (for instance because we use box content) you need to use the
alternatives \type {node.copy_list} and \type {node.free_list} instead.

In \CONTEXT\ there is a convenient helper to create a list of text nodes:

\startbuffer
\startluacode
context(nodes.typesetters.tonodes("this works okay"))
\stopluacode
\stopbuffer

\typebuffer

And indeed, \getbuffer, even when we use spaces. Of course it makes
more sense (and it is also more efficient) to do this:

\startbuffer
\startluacode
context("this works okay")
\stopluacode
\stopbuffer

In this case the list is constructed at the \TEX\ end. We have now learned enough
to start using some convenient operations, so these are introduced next. Instead
of the longer \type {tonodes} call we will use the shorter one:

\starttyping
local head, tail = string.tonodes("this also works"))
\stoptyping

As you see, this constructor returns the head as well as the tail of the
constructed list.

\stopsection

\startsection[title=Operations]

If you are familiar with \LUA\ you will recognize this kind of code:

\starttyping
local str = "time: " .. os.time()
\stoptyping

Here a string \type {str} is created that is built out if two concatinated
snippets. And, \LUA\ is clever enough to see that it has to convert the number to
a string.

In \CONTEXT\ we can do the same with nodes:

\startbuffer
\startluacode
local foo = string.tonodes("foo")
local bar = string.tonodes("bar")
local amp = string.tonodes(" & ")

context(foo .. amp .. bar)
\stopluacode
\stopbuffer

\typebuffer

This will append the two node lists: \getbuffer.

\startbuffer
\startluacode
local l = string.tonodes("l")
local m = string.tonodes(" ")
local r = string.tonodes("r")

context(5 * l .. m .. r * 5)
\stopluacode
\stopbuffer

\typebuffer

You can have the multiplier on either side of the node: \getbuffer.
Addition and subtraction is also supported but it comes in flavors:

\startbuffer
\startluacode
local l1 = string.tonodes("aaaaaa")
local r1 = string.tonodes("bbbbbb")
local l2 = string.tonodes("cccccc")
local r2 = string.tonodes("dddddd")
local m  = string.tonodes(" + ")

context((l1 - r1) .. m .. (l2 + r2))
\stopluacode
\stopbuffer

\typebuffer

In this case, as we have two node (lists) involved in the addition and
subtraction, we get one of them injected into the other: after the first, or
before the last node. This might sound weird but it happens.

\dontleavehmode \start \maincolor \getbuffer \stop

We can use these operators to take a slice of the given node list.

\startbuffer
\startluacode
local l = string.tonodes("123456")
local r = string.tonodes("123456")
local m = string.tonodes("+ & +")

context((l - 3) .. (1 + m - 1).. (3 + r))
\stopluacode
\stopbuffer

\typebuffer

So we get snippets that get appended: \getbuffer. The unary operator
reverses the list:

\startbuffer
\startluacode
local l = string.tonodes("123456")
local r = string.tonodes("123456")
local m = string.tonodes(" & ")

context(l .. m .. - r)
\stopluacode
\stopbuffer

\typebuffer

This is probably not that useful, but it works as expected: \getbuffer.

We saw that \type {*} makes copies but sometimes that is not enough. Consider the
following:

\startbuffer
\startluacode
local n = string.tonodes("123456")

context((n - 2) .. (2 + n))
\stopluacode
\stopbuffer

\typebuffer

Because the slicer frees the unused nodes, the value of \type {n} in the second
case is undefined. It still points to a node but that one already has been freed.
So you get an error message. But of course (as already demonstrated) this is
valid:

\startbuffer
\startluacode
local n = string.tonodes("123456")

context(2 + n - 2)
\stopluacode
\stopbuffer

\typebuffer

We get the two middle characters: \getbuffer. So, how can we use a
node (list) several times in an expression? Here is an example

\startbuffer
\startluacode
local l = string.tonodes("123")
local m = string.tonodes(" & ")
local r = string.tonodes("456")

context((l^1 .. r^1)^2 .. m^1 .. r .. m .. l)
\stopluacode
\stopbuffer

\typebuffer

Using \type {^} we create copies, so we can still use the original later on. You
can best make sure that one reference to a node is not copied because otherwise
we get a memory leak. When you write the above without copying \LUATEX\ most
likely end up in a loop. The result of the above is:

\blank \start \dontleavehmode \maincolor \getbuffer \stop \blank

Let's repeat it once more time: keep in mind that we need to do the memory
management ourselves. In practice we will seldom need more than the
concatination, but if you make complex expressions be prepared to loose some
memory when you copy and don't free them. As \TEX\ runs are normally limited in
time this is hardly an issue.

So what about the division. We needed some kind of escape and as with \type
{lpeg} we use the \type {/} to apply additional operations.

\startbuffer
\startluacode
local l = string.tonodes("123")
local m = string.tonodes(" & ")
local r = string.tonodes("456")

local function action(n)
    for g in node.traverse_id(node.id("glyph"),n) do
        g.char = string.byte("!")
    end
    return n
end

context(l .. m / action .. r)
\stopluacode
\stopbuffer

\typebuffer

And indeed we the middle glyph gets replaced: \getbuffer.

\startbuffer
\startluacode
local l = string.tonodes("123")
local r = string.tonodes("456")

context(l .. nil .. r)
\stopluacode
\stopbuffer

\typebuffer

When you construct lists programmatically it can happen that one of the
components is nil and to some extend this is supported: so the above
gives: \getbuffer.

Here is a summary of the operators that are currently supported. Keep in mind that
these are not built in \LUATEX\ but extensions in \MKIV. After all, there are many
ways to map operators on actions and this is just one.

\starttabulate[|l|l|]
\NC \type{n1 .. n2} \NC append nodes (lists) \type {n1} and \type {n2}, no copies \NC \NR
\NC \type{n * 5}    \NC append 4 copies of node (list) \type {n} to \type {n} \NC \NR
\NC \type{5 + n}    \NC discard the first 5 nodes from list \type {n} \NC \NR
\NC \type{n - 5}    \NC discard the last 5 nodes from list \type {n} \NC \NR
\NC \type{n1 + n2}  \NC inject (list) \type {n2} after first of list \type {n1} \NC \NR
\NC \type{n1 - n2}  \NC inject (list) \type {n2} before last of list \type {n1} \NC \NR
\NC \type{n^2}      \NC make two copies of node (list) \type {n} and keep the orginal \NC \NR
\NC \type{- n}      \NC reverse node (list) \type {n} \NC \NR
\NC \type{n / f}    \NC apply function \type {f} to node (list) \type {n} \NC \NR
\stoptabulate

As mentioned, you can only use a node or list once, so when you need it more times, you need
to make copies. For example:

\startbuffer
\startluacode
local l = string.tonodes(     -- maybe: nodes.maketext
    " 1 2 3 "
)
local r = nodes.tracers.rule( -- not really a user helper (spec might change)
    string.todimen("1%"),     -- or maybe: nodes.makerule("1%",...)
    string.todimen("2ex"),
    string.todimen(".5ex"),
    "maincolor"
)

context(30 * (r^1 .. l) .. r)
\stopluacode
\stopbuffer

\typebuffer

This gives a mix of glyphs, glue and rules: \getbuffer. Of course you can wonder
how often this kind of juggling happens in use cases but at least in some core
code the concatination (\type {..}) gives a bit more readable code and the
overhead is quite acceptable.

\stopsection

\stopchapter

\stopcomponent
