% language=us

\environment interaction-style

\startcomponent interaction-introduction

\startchapter[title=Introduction]

This document introduces the cross reference mechanism, viewer control, fill||in
fields, \JAVASCRIPT\ support, comments, attachments and more. It is a rewrite of
the \MKII\ widgets manual. There is (always) more available than discussed in
manuals so if you miss something, take a look at test suite or when you're brave,
peek into the source code as there can be examples there.

Interactivity has always been available in \CONTEXT\ and in fact it was one of
the reasons for writing it. In for instance the YandY \WINDOWS\ previewer, one
could have hyperlinks and we used that for a while when checking documents. Later
Acrobat showed up and \PDF\ stepwise added interactive features that we always
supported right from the start. Unfortunately there is a viewer dependency and
the documentation of \PDF\ lagged behind, so solutions based on trial and error
could not work well in a follow up on \PDF. Some features disappeared or became
so limited that they effectively became useless. Especially multi||media have a
reputation of unreliability. Because open source viewers never really catched up
(at least not in this area) the momentum was lost to make sure that documents
could have audio and video embedded in reliable ways. Even forms and basic
\JAVASCRIPT\ control of for instance layers is often missing.

That said, we do support a lot but can support more when it makes sense. Deep
down in \CONTEXT\ we always had the mechanisms to deal with this, so extensions
are not that hard to program. Somehow we thought that publishers would like these
features but that never really was the case, so there was no pressure from that
end. Most features are user driven or just there because at some point we wanted
to make some fancy presentation. In fact, the \type {s-present-*} files provide
examples of interactivity.

The original \PDF\ reference was a couple of hundred pages and looked quite nice.
A later print has many more pages and still looks ok, but nowadays we have to do
with a \PDF\ document. If you want to see what \PDF\ supports you can study this
(now about) 750 page standard. It is, being an \ISO\ standard, not public but
you can probably find a (maybe older) copy someplace on the web.

When reading this manual you need to keep in mind that we assume that you design
a decent layout and when you make something for an electronic medium, we hope
that you pay attention to the way you can enhance accessibility.

If you miss something here, don't hesitate to ask for clarification, or even
better, provide an example that we then can use to discuss (an aspect of) some
mechanism.

\stopchapter

\stopcomponent

% A nice double page e-ink device can be seen at:
%
% https://www.youtube.com/watch?v=QdOXCn1vvzI :
%
% vkgoeswild: Playing Pink Floyds Great Gig in the Sky on Imperial by Bösendorfer
