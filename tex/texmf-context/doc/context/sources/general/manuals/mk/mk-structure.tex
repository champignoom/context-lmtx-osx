% language=us

\usemodule[narrowtt]

\environment mk-environment

\startcomponent mk-structure

\chapter{Everything structure}

At the time of this writing, \CONTEXT\ \MKIV\ spends some 50\% of
its time in \LUA. There are several reasons for this.

\startitemize[packed]
\item All \IO\ goes via \LUA, including messages and logging. This includes
      file searching which happened to be done by the \KPSE\ library.
\item Much font handling is done by \LUA\ too, for instance \OPENTYPE\ features
      are completely handled by \LUA.
\item Because \TEX\ is highy optimized, its influence on runtime is less
      prominent. Even if we delegate some tasks to \LUA, \TEX\ still has
      work to do.
\stopitemize

Among the reported statistics of a 242 page version of \type
{mk.pdf} (not containing this chapter) we find the following:

\startntyping
input load time           - 0.094 seconds
startup time              - 0.905 seconds (including runtime option file processing)
jobdata time              - 0.140 seconds saving, 0.062 seconds loading
fonts load time           - 5.413 seconds
xml load time             - 0.000 seconds, lpath calls: 46, cached calls: 31
lxml load time            - 0.000 seconds preparation, backreferences: 0
mps conversion time       - 0.000 seconds
node processing time      - 1.747 seconds including kernel
kernel processing time    - 0.343 seconds
attribute processing time - 2.075 seconds
language load time        - 0.109 seconds, n=4
graphics processing time  - 0.109 seconds including tex, n=7
metapost processing time  - 0.484 seconds, loading: 0.016 seconds, execution: 0.203 seconds, n: 65
current memory usage      - 332 MB
loaded patterns           - gb:gb:pat:exc:3 nl:nl:pat:exc:4 us:us:pat:exc:2
control sequences         - 34245 of 165536
callbacks                 - direct: 235579, indirect: 18665, total: 254244 (1050 per page)
runtime                   - 25.818 seconds, 242 processed pages, 242 shipped pages, 9.373 pages/second
\stopntyping

The startup time includes initial font loading (we don't store fonts
in the format). Jobdata time involves loading and saving multipass data
used for tables of contents, references, positioning, etc. The time needed
for loading fonts is over 5 seconds due to the fact that we load a couple of
real large and complex fonts. Node processing time mostly is related to
\OPENTYPE\ feature support. The kernel processing time refers to hyphenation
and line breaking, for which (of course) we use \TEX. Direct callbacks are
implicit calls to \LUA, using \type {\directlua} while the indirect calls
concern overloaded \TEX\ functions and callbacks triggered by \TEX\ itself.

Depending on the system load on my laptop, the throughput is around
10 pages per second for this document, which is due to the fact
that some font trickery takes place using a few arabic fonts, some
chinese, a bunch of metapost punk instances, Zapfino, etc.

The times reported are accumulated times and contain quite some
accumulated rounding errors so assuming that the operating system
rounds up the times, the totals in practice might be higher. So,
looking at the numbers, you might wonder if the load on \LUA\ will
become even larger. This is not necessary. Some tasks can be done
better in \LUA\ but not always with less code, especially when we
want to extend functionality and to provide more robust solutions.
Also, even if we win some processing time we might as well waste
it in interfacing between \TEX\ and \LUA. For instance, we can
delegate pretty printing to \LUA, but most documents don't contain
verbatim at all. We can handle section management by \LUA, but how
many section headers does a document have?

When the future of \TEX\ is discussed, among the ideas presented
is to let \TEX\ stick to typesetting and implement it as a
component (or library) on top of a (maybe dedicated) language.
This might sound like a nice idea, but eventually we will end up
with some kind of user interface and a substantial amount of code
dedicated to dealing with fonts, structure, character management,
math etc.

In the process of converting \CONTEXT\ to \MKIV\ we try to use
each language (\TEX, \LUA, \METAPOST) for what it is best suited
for. Instead of starting from scratch, we start with existing code
and functionality, because we need a running system. Eventually we
might find \TEX's role as language being reduced to (or maybe we can
better talk of \quote {focused on}) mostly aspects of
typesetting, but \CONTEXT\ as a whole will not be much different
from the perspective of the user.

So, this is how the transition of \CONTEXT\ takes place:

\startitemize[packed]
\item We started with replacing isolated bits and pieces of code
      where \LUA\ is a more natural candidate, like file \IO, encoding
      issues.
\item We implement new functionality, for instance \OPENTYPE\
      and \TYPEONE\ support.
\item We reimplement mechanisms that are not efficient as we want them
      to be, like buffers and verbatim.
\item We add new features, for instance tree based \XML\ processing.
\item After evaluating we reimplement again when needed (or when \LUATEX\
      evolves).
\stopitemize

Yet another transition is the one we will discuss next:

\startitemize[packed]
\item We replace complex mechanisms by new ones where we separate
      management and typesetting.
\stopitemize

This not so trivial effort because it affects many aspects of \CONTEXT\ and
as such we need to adapt a lot of code at the same time: all things
related to structure:

\startitemize[packed]
\item sectioning (chapters, sections, etc)
\item numbering (pages, itemize, enumeration, floats, etc)
\item marks (used for headers and footers)
\item lists (tables of contents, lists of floats, sorted lists)
\item registers (including collapsing of page ranges)
\item cross referencing (to text as well as pages)
\item notes (footnotes, endnotes, etc)
\stopitemize

All these mechanisms are somehow related. A section head can occur
in a list, can be cross referenced, might be shows in a header and
of course can have a number. Such a number can have multiple
components (1.A.3) where each component can have its own
conversion, rendering (fonts, colors) and selectively have less
components. In tables of contents either or not we want to see all
components, separators etc. Such a table can be generated at each
level, which demands filtering mechanisms. The same is true for
registers. There we have page numbers too, and these may be
prefixed by section numbers, possibly rendered differently than
the original section number.

Much if this is possible in \CONTEXT\ \MKII, but the code that
deals with this is not always nice and clean and right from the start
of the \LUATEX\ project it has been on the agenda to clean it up. The code
evolved over time and
functionality was added when needed. But, the projects
that we deal with demand more (often local) control over the
components of a number.

What makes structure related data complex is that we need to keep
track of each aspect in order to be able to reproduce the
rendering in for instance a table of contents, where we also may
want to change some of the aspects (for instance separators in a
different color). Another pending issue is \XML\ and although we
could normally deal with this quite well, it started making sense
to make all multi|-|pass data (registers, tables of content,
sorted lists, references, etc.) more \XML\ aware. This is a
somewhat hairy task, if only because we need to switch between
\TEX\ mode and \XML\ mode when needed and at the same time keep an
eye on unwanted expansion: do we keep structure in the content or
not?

Rewriting the code that deals with these aspects of typesetting is
the first step in a separation of code in \MKII\ and \MKIV. Until
now we tried to share much code, but this no longer makes sense.
Also, at the \CONTEXT\ conference in Bohinj (2008) it was decided
that given the development of \MKIV, it made sense to freeze
\MKII\ (apart from bug fixes and minor extensions). This decision
opens the road to more drastic changes. We will roll back some of
the splits in code that made sharing code possible and just
replace whole components of \CONTEXT\ as a whole. This also gives
us the opportunity to review code more drastically than until now
in the perspective of \ETEX.

Because this stage in the rewrite of \CONTEXT\ might bring some
compatibility issues with it (especially for users who use the
more obscure tuning options), I will discuss some of the changes
here. A bit of understanding might make users more tolerant.

The core data structure that we need to deal with is a number, which
can be constructed in several ways.

\def\NotaBeneR{\inframed[frame=off,background=color,backgroundcolor=mktransparentred]}
\def\NotaBeneG{\inframed[frame=off,background=color,backgroundcolor=mktransparentgreen]}
\def\NotaBeneB{\inframed[frame=off,background=color,backgroundcolor=mktransparentblue]}
\def\NotaBeneY{\inframed[frame=off,background=color,backgroundcolor=mktransparentyellow]}
\def\NotaBeneS{\inframed[frame=off,background=color,backgroundcolor=mktransparentgray]}

\starttabulate[|l|l|]
\NC sectioning   \NC \NotaBeneR{1.A.2.II} some title \NC \NR
\NC pagenumber   \NC page \NotaBeneR{1.A}\NotaBeneG{--}\NotaBeneB{23} \NC \NR
\NC reference    \NC in chapter \NotaBeneR{2.II} \NC \NR
\NC marking      \NC \NotaBeneR{A}: some title with preceding number \NC \NR
\NC contents     \NC \NotaBeneR{2.II} some title with some page number \NotaBeneR{1.A}\NotaBeneG{--}\NotaBeneB{23} \NC \NR
\NC index        \NC some word \NotaBeneB{23}, \NotaBeneR{A}\NotaBeneG{--}\NotaBeneB{42}---\NotaBeneR{B}\NotaBeneG{--}\NotaBeneB{48} \NC \NR
\NC itemize      \NC \NotaBeneY{a} first item \NotaBeneY{a.1} subitem item \NC \NR
\NC enumerate    \NC example \NotaBeneR{1.A.2.II}\NotaBeneG{.}\NotaBeneY{a} \NC \NR
\NC floatcaption \NC figure \NotaBeneR{1}\NotaBeneG{--}\NotaBeneB{2} \NC \NR
\NC footnotes    \NC note \NotaBeneS{\symbol[3]} \NC \NR
\stoptabulate

In this table we see how numbers are composed:

\starttabulate[|l|p|]
\NC \NotaBeneR{section number} \NC It has several components, separated by symbols
                                   and with an optional final symbol \NC \NR
\NC \NotaBeneG{separator}      \NC This can be different for each level and can
                                   have dedicated rendering options \NC \NR
\NC \NotaBeneB{page number}    \NC That can be preceded by a (partial) sectionnumber
                                   and separated from the page number by another symbol \NC \NR
\NC \NotaBeneY{counter}        \NC It can be preceded by a (partial) sectionnumber and
                                   can also have subnumbers with its own separation
                                   properties \NC \NR
\NC \NotaBeneS{symbol}         \NC Sometimes numbers get represented by symbols in which
                                   case we use pagewise restarting symbol sets \NC \NR
\stoptabulate

Say that at some point we store a section number and/or page
number. With the number we need to store information about the
conversion (number, character, roman numeral, etc) and the
separators, including their rendering. However, when we reuse that
stored information we might want to discard some components and/or
use a different rendering. In traditional \CONTEXT\ we have
control over some aspects but due to the way numbers are stored
for later reuse this control is limited.

Say that we have cloned a subsection head as follows:

\starttyping
\definehead[MyHead][section]
\stoptyping

This is used as:

\starttyping
\MyHead[example]{Example}
\stoptyping

In \MKII\ we save a list entry (which has the number, the title
and a reference to the page) and a reference to the the number,
the title and the page (tagged \type {example}). Page numbers are
stored in such a way that we can filter at specific section
levels. This permits local tables of contents.

The entry in the multi pass data file looks as follows (we collect all
multi pass data in one file):

\starttyping
\mainreference{}{example}{2--0-1-1-0-0-0-0--1}{1}{{I.I}{Example}}%
\listentry{MyHead}{2}{I.I}{Example}{2--0-1-1-0-0-0-0--1}{1}%
\stoptyping

In \MKIV\ we store more information and use tables for that. Currently
the entry looks as follows:

\starttyping
structure.lists.collected={
 {
   ...
 },
 {
  metadata={
   catcodes=4,
   coding="tex",
   internal=2,
   kind="section",
   name="MyHead",
   reference="example",
  },
  pagenumber={
   numbers={ 1, 1, 0 },
  },
  sectionnumber={
   conversion="R",
   conversionset="default",
   numbers={ 0, 2 },
   separatorset="default",
  },
  sectiontitle={
   label="MyHead",
   title="Example",
  },
 },
 {
  ...
 },
}
\stoptyping

There can be much more information in each of the subtables. For
instance, the \type {pagenumber} and \type {sectionnumber}
subtables can have \type {prefix}, \type {separatorset},
\type{conversion}, \type {conversionset}, \type {stopper}, \type
{segments} and \type {connector} fields, and the \type {metadata}
table can contain information about the \XML\ root document so
that associated filtering and handling can be reconstructed. With the
section title we store information about the preceding label text
(seldom used, think of \quote{Part B}).

This entry is used for lists as well as cross referencing.
Actually, the stored information is also used for markings
(running heads). This means that these mechanisms must be able to
distinguish between where and how information is stored.

These tables look rather verbose and indeed they are. We end up
with much larger multi|-|pass data files but fortunately loading them
is quite efficient. Serializing on the other hand might cost some time
which is compensated by the fact that we no longer store
information in token lists associated with nodes in \TEX's lists
and in the future we might even move more data handling to the
\LUA\ end. Also, in future versions we will share similar data
(like page number information) more efficiently.

Storing date at the \LUA\ end also has consequences for the
typesetting. When specific data is needed a call to \LUA\ is
necessary. In the future we might offer both push and pull methods
(\LUA\ pushing information to the typesetting code versus \LUA\
triggering typesetting code). For lists we pull, and for registers
we currently push. Depending on our experiences we might change
these strategies.

A side effect of the rewrite is that we force more consistency.
For instance, you see a \type {conversion} field in the list. This
is the old way of defining the way a number gets converted. The
modern approach is to use sets. Because we now have a more
stringent inheritance model at the user interface level, this
might lead to incompatible conversions at lower levels (when
unset). Instead of cooking up some nasty compatibility hacks, we
accept some incompatibility, if only because users have to adapt
their styles to new font technology anyway. And for older
documents there is still \MKII.

Instead of introducing many extra configuration variables (for each
level of sectioning) we introduce sets. These replace some of the
existing parameters and are the follow up on some (undocumented)
precursor of sets. Examples of sets are:

\starttyping
\definestructureseparatorset [default][][.]
\definestructureconversionset[default][][numbers]
\definestructureresetset     [default][][0]
\definestructureprefixset    [default][section-2,section-3][]
\definestructureseparatorset [appendix][][.]
\definestructureconversionset[appendix][Romannumerals,Characters][]
\definestructureresetset     [appendix][][0]
\stoptyping

The third parameter is the default value. The sets that relate to typesetting
can have a rendering specification:

\starttyping
\definestructureseparatorset
  [demosep]
  [demo->!,demo->?,demo->*,demo->@]
  [demo->/]
\stoptyping

Here we apply \type{demo} to each of the separators as well as to the
default. The renderer is defined with:

\starttyping
\defineprocessor[demo][style=\bfb,color=red]
\stoptyping

You can imagine that, although this is quite possible in \TEX,
dealing with sets, splitting them, handling the rendering, etc.\
is easier in \LUA\ that in \TEX. Of course the code still looks
somewhat messy, if only because the problem is messy. Part if this
mess is related to the fact that we might have to specify all
components that make up a number.

\starttabulate
\NC section    \NC section number as part of head  \NC \NR
\NC list       \NC section number as part of list entry  \NC \NR
\NC            \NC section number as part of page number prefix \NC \NR
\NC            \NC (optionally prefixed) page number \NC \NR
\NC counter    \NC section number as part of counter prefix  \NC \NR
\NC            \NC (optionally prefixed) counter value(s) \NC \NR
\NC pagenumber \NC section number as part of page number \NC \NR
\NC            \NC pagenumber components (realpage, page, subpage) \NC \NR
\stoptabulate

As a result we have upto 3 sets of parameters:

\starttabulate
\NC section    \NC \type{section*} \NC \NR
\NC list       \NC \type{section*} \type{prefix*} \type{page*} \NC \NR
\NC counter    \NC \type{section*} \type{number*} \NC \NR
\NC pagenumber \NC \type{prefix*} \type{page*} \NC \NR
\stoptabulate

When reimplementing the structure related commands, we also have
to take mechanisms into account that relate to them. For instance,
index sorter code is also used for sorted lists, so when we adapt
one mechanism we also have to adapt the other. The same is true
for cross references, that are used all over the place. It helps
that for the moment we can omit the more obscure interaction
related mechanism, if only because users will seldom use them.
Such mechanisms are also related to the backend and we're not yet
in the stage where we upgrade the backend code. In case you wonder
why references can be such a problematic areas think of the
following:

\starttyping
\goto{here}[page(10),StartSound{ping},StartVideo{demo}]
\goto{there}[page(10),VideLayer{example},JS(SomeScript{hi world})]
\goto{anywhere}[url(mypreviouslydefinedurl)]
\stoptyping

The \CONTEXT\ cross reference mechanism permits mixed usage of simple
hyperlinks (jump to some page) and more advanced viewer actions like
showing widgets and runnign \JAVASCRIPT\ code. And even a simple
reference like:

\starttyping
\at{here and there}[somefile::sometarget]
\stoptyping

involves some code because we need to handle the three words as
well as the outer reference. \footnote {Currently \CONTEXT\ does
its own splitting of multiword references, and does so by reusing
hyperlink resources in the backend format. This might change in
the future.} The reason why we need to reimplement referencing
along with structure lays in the fact that for some structure
components (like section headers and float references) we no
longer store cross reference information separately but filter it
from the data stored in the list (see example before).

The \LUA\ code involved in dealing with the more complex
references shown here is much more flexible and robust than the
original \TEX\ code. This is a typical example of where the
accumulated time spent on the \TEX\ based solution is large
compared to the time spent on the \LUA\ variant. It's like driving
200 km by car through hilly terrain and wondering how one did that
in earlier times. Just like today scenery is not by definition better
than yestedays, \MKIV\ code is not always better than \MKII\ code.

\stopcomponent
