%D \module
%D   [      file=t-embossed,
%D        version=2007.07.17, 
%D          title=\CONTEXT\ Style File,
%D       subtitle=Presentation Module embossed,
%D         author=Thomas A. Schmitz,
%D           date=\currentdate,
%D      copyright={Thomas A. Schmitz}]
%C
%C Copyright 2007 Thomas A. Schmitz.
%C This file may be distributed under the GNU General Public License v. 2.0.

%D This file provides the \quotation{embossed} style for the presentation
%D module. It is loaded at runtime. 

\writestatus{loading}{module embossed}

\startmodule[embossed]

\unprotect

%D The taspresentation module provides a skeleton into which different styles
%D can be hooked. It uses a number of variables and macros which have to be set
%D beforehand. Some parts are optional. We begin with the necessary definitions:

%D We start colors:

\setupcolors[state=start]

%D These macros are used for placing figures/pictures:

\define\NormalHeight{.94\textheight}
\define\NormalWidth{.476\textwidth}
\define\PictureFrameHeight{.94\textheight}
\define\PictureFrameWidth{.476\textwidth}

%D The page layout:

\setuplayout [width=fit,
              margin=1.3cm,
              height=fit,
              header=1cm, 
              footer=1cm, 
              topspace=10mm, 
              backspace=2cm,
              location=singlesided]

%D The macro for typesetting the Slidetitle; this is adapted from a sample
%D document that Brooks Moses published on the wiki:

\definelayer[slidetitle]
    [width=\paperwidth,
    height=\paperheight,
    x=20mm,
    y=2mm]

\define[1]\Slidetitle{\page\setlayer[slidetitle]%
     {\framed[frame=off,height=2cm,width=\textwidth,offset=0pt,align=middle,top=\vss,bottom=\vss]{\switchtobodyfont[\Titlesize]\color[e]{#1}}}}

%D The macro \tex{Maketitle} produces a default title page with the author, the
%D title of the presentation, and the date. Using it is not mandatory.

\define\Maketitle{%
\titback
\null
\vfill
\midaligned{\switchtobodyfont[\Titlesize]\color[e]{\getvariable{taspresent}{title}}}
\blank[2*line]
\midaligned{\getvariable{taspresent}{author}}
\blank[3*line]
\midaligned{\currentdate}
\vfill
\null}

%D The following parts are optional; if you don't use backgrounds and are
%D content with CONTEXT's default itemization, you don't have to set these
%D macros. 

%D We define our colors:

\definecolor [a]	       [r=1,g=1,b=.8]
\definecolor [Item]            [r=.2,b=.2,b=.5]
\definecolor [b]               [r=.6,g=.2,b=.2]
\definecolor [c] 	       [r=.7,g=.2,b=.2]
\definecolor [d] 	       [r=.4,g=.2,b=.2]
\definecolor [e] 	       [r=.2,g=.2,b=.5]
\definecolor [f]               [s=.4]

\beginOLDTEX
\definefontsynonym [BigBoss] [uhvb8r] 
\definefont[Emblem] [BigBoss at 30pt]
\endOLDTEX

\beginLUATEX
\definefontsynonym [BigBoss] [name:TeXGyreHeros-Bold]
\definefont[Emblem] [BigBoss at 30pt]
\endLUATEX

%D However, it seemed more portable to let Metapost calculate the background.
%D The rainbow effect takes some lines of code, but it's not too bad:

\startuseMPgraphic{redyellow}
StartPage ;
picture boss ;
picture bboss ;
pair zf[] ;
fill Page withcolor \MPcolor{a} ;
path gr[] ;
numeric a; a = 2cm ;
numeric b; b = 0.9cm ;
z.f1 = llcorner Page shifted (0,a) ;
z.f2 = lrcorner Page shifted (0,a) ;
gr[1] = llcorner Page -- z.f1 -- z.f2 -- lrcorner Page -- cycle ;
fill gr[1] withcolor \MPcolor{b} ;
pickup pencircle scaled 6pt ;
gr[2] = ulcorner Page -- urcorner Page -- lrcorner Page -- llcorner Page -- cycle ;
gr[2] := gr[2] enlarged -2.8.pt ;
draw gr[2] withcolor \MPcolor{f} ;
draw z.f1 -- z.f2 withcolor \MPcolor{f} ;
boss := \sometxt{\framed[width=\textwidth,height=2cm,offset=0pt,align=right,top=\vss,bottom=\vss,frame=off]{\Emblem \color[c]{Made with \CONTEXT\hfill \pagenumber\ of \lastpage}}} ;
bboss := \sometxt{\framed[width=\textwidth,height=2cm,offset=0pt,align=right,top=\vss,bottom=\vss,frame=off]{\strut\Emblem \color[d]{Made with \CONTEXT\hfill \pagenumber\ of \lastpage}}} ;
draw bboss shifted (1.96cm,0.04cm) ;
draw boss shifted (2cm,0) ; 
StopPage ;
\stopuseMPgraphic


%D We define these backgrounds as overlays:

\defineoverlay[yellowbg]
[\useMPgraphic{redyellow}]

%D These are shortcuts to switch backgrounds:

\define\lecback{\setuplayout[header=10mm]\setupbackgrounds[page][background={yellowbg,slidetitle}]}
\define\titback{\setuplayout[header=0mm]\setupbackgrounds[page][background={yellowbg}]}
\define\picback{\setuplayout[header=0mm]\setupbackgrounds[page][background={yellowbg}]}
\define\noback{\setupbackgrounds[page][background=nobackground]}

%D We use combinations for placing vertical pictures and text side by side, and
%D we want a distance of 1.1 cm between both.

\setupcombinations[distance=1.1cm]

%D The symbol for the first level of itemizations. 

\definesymbol[1][\useMPgraphic{ItSquare}]
\setupitemize[1][inmargin][color=e]

\protect
\stopmodule

\endinput

