\usemodule[rst][test=yes]
\setuphead[chapter][page=no,style=bold]

\def\RSTCTX{{\em rst}\kern.5pt\CONTEXT}
\def\reST{{\rm re}{\ss Structured}{\rm Text}}

\starttext

\chapter{\RSTCTX\ Hybrid Documents}

This example demonstrates the macro \type{\RST} which can be used
to process \reST\ markup directly in a normal \CONTEXT\ document.

\RST{
-------------
This Chapter,
-------------

… for instance, was given entirely in *reST* markup. Naturally,
there are some :bold:`drawbacks` to expect when mixing markups:
directives and hyperlink targets that have already been specified
somewhere above the current section will *stay* accessible in
later passages until you redefine them. Also, certain letters
need to be thoroughly escaped in order for them to make it
through to the *reST*-parser, e.g. *\\\{* (<left brace>), and
you’ll have to be inventive to make a backslash
(*\\letterbackslash*) pass through the parser.

}

\startRST

-------------
Alternatively
-------------
you may always use the matching environment ``\\[start|stop]RST``
if you prefer.

\stopRST


\stoptext
