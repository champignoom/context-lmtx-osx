\usemodule[taspresent][style=rules,font=MyriadPro,size=17pt,stylecolor=red]

\setvariables [taspresent]
              [author={Groucho Marx},
               title={Marriage the Chief Cause of Divorce}]

% \usetypescriptfile[type-gtamacfonts]
% \definetypeface[MyFace][ss][sans][optima][default][encoding=ec]
% %\usetypescript[LucidaSans][ec]
% \setupbodyfont[MyFace,ss,\Normalsize]

% \usesymbols[lucidasans]
% \setupsymbolset[LucidaSans]
% \definesymbol[1][{\symbol[Check2]}]

% \showframe

% \traceexternalfigurestrue

\starttext

\Maketitle

\Slidetitle{Text}

\lecback

Thus, I came to the conclusion that the designer of a new
system must not only be the implementer and first
large||scale user; the designer should also write the first
user manual.

The separation of any of these four components would have
hurt \TeX\ significantly. If I had not participated fully in
all these activities, literally hundreds of improvements
would never have been made, because I would never have
thought of them or perceived why they were important.

But a system cannot be successful if it is too strongly
influenced by a single person. Once the initial design is
complete and fairly robust, the real test begins as people
with many different viewpoints undertake their own
experiments.

\Slidetitle{Itemization}

\startitemize[1]
\item Thus, I came to the conclusion that the designer of a new
	system  
\item must not only be the implementer and first
	large||scale user;  
\item the designer should also write the first
	user manual.
\item The separation of any of these four components would have
	hurt \TeX\ significantly. 
\stopitemize

\Slidetitle{Numbered Itemization}

\startitemize[n]
\item Thus, I came to the conclusion that the designer of a new
	system  
\item must not only be the implementer and first
	large||scale user;  
\item the designer should also write the first
	user manual.
\item The separation of any of these four components would have
	hurt \TeX\ significantly. 
\stopitemize

\Slidetitle{Picture in Horizontal Mode}

\PicHoriz[hor][height=\NormalHeight]

\page

\picback 

\PicVert[vert][width=\NormalWidth]{Picture in \\ Vertical Mode}

\page

\CircVert[scale=22,x=23,y=25][vert][width=\NormalWidth]{Circle in \\ Vertical Mode}

\page

\ArrowVert[direction=90,x=7,y=23][vert][width=\NormalWidth]{Arrow in \\ Vertical Mode}

\Slidetitle{Red Circle}

\lecback

\CircHoriz[scale=40,x=120,y=80][hor][height=\NormalHeight]

\Slidetitle{Red Arrow}

\ArrowHoriz[direction=135,x=105,y=15][hor][height=\NormalHeight]

\Slidetitle{A MetaFun graphic}

\placefigure[here]{none}{%
\startMPcode
pickup pencircle scaled 4pt ;
draw unitsquare xyscaled (5cm,5cm) withcolor red ;
\stopMPcode
}

\Slidetitle{Some Code Snippets}

To set up a horizontal picture, simply type:

\startTEX
\PicHoriz[hor][height=\Normalheight]
\stopTEX

\blank[line]

For vertical pictures:

\startTEX
\PicVert[vert][width=\NormalWidth]%
{Text placed \\ opposite picture}
\stopTEX

\Slidetitle{Math}

Since I know nothing about math, this example is copied from the wiki:

\startformula
 f(x) = \startmathcases
   \NC x, \NC if $0 \le x \le \frac12$ \NR
   \NC 1-x ,\NC if $\frac12 \le x \le 1$ \NR
\stopmathcases
\stopformula

\Slidetitle{Your Own Ideas?}

\null

\vfill

\midaligned{\tfd Go \color[red]{here!}}

\vfill

\page

\stoptext

%%% Local Variables: 
%%% mode: context
%%% TeX-master: t
%%% End: 
